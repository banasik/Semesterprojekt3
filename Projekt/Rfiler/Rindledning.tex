
\chapter{Indledning}
I dette projekt arbejdes der med blodtryksmålere. Vi har valgt at udarbejde en blodtryksmåler til forskningsbrug. Blodtryksmåleren skal kunne modtage en spænding fra en transducer, nulpunktsjustere og kalibere efter en forskers ønske. Signalet skal vises i en graf, på et display, hvor værdier for puls, systoliske- og diastoliske tryk vises. Her fra forskeren starter og gemmer sine målinger.\\
I kravspecifikationen findes de krav, der er blevet stillet for projektet. Herunder er også de krav, som blev stillet mellem os og vores vejleder.\\
Under systemarkitekturen findes informationer om, hvordan software- og hardwaredelen er opbygget.  I afsnittet integrationstest kan der læses om, hvordan projektet er blevet testet.\\  

\subsubsection{Versionshistorik}

\begin{longtabu} to \linewidth{@{}l l l X[j]@{}}
    Version &    Dato &    Ansvarlig &    Beskrivelse\\[-1ex]
    \midrule
    0.1 &   04-11-2015	&   MHNK  &   Oprettelse af LaTex dokumenter \\
    0.2 &   11-11-2015	&   ABH  &   Udviklingsværktøjer og krav \\
    0.3 &   11-11-2015	&   DHC  &   Metoder \\
    0.4 &   18-11-2015	&   ABH  &   Systembeskrivelse  \\
    0.5 &   19-11-2015	&   DHC  &   Perspektivering - Fremtidigt arbejde \\
    0.6 &   24-11-2015	&   MHNK  &   Blodtryk \\
    0.7 &   24-11-2015	&   AJF  &   Projektgennemførelse og -styring \\
    0.8 &   24-11-2015	&   MM  &   Projektformulering og afgrænsning \\
    1.0 &   01-12-2015	&   MB, MHNK  &   Referencelister i LaTex \\
    1.1 &   02-12-2015	&   MB, MHNK  &   Figurlister i LaTex \\
    1.2 &   02-12-2015	&   MHNK  &   Resume \\
    1.3 &   07-12-2015	&   MHNK  &   Abstract \\
    1.4 &   09-12-2015	&   MHNK  &   Konklusion \\
    1.5 &   11-12-2015	&   ABH  &   Tilrette krav og systembeskrivelse ift. endelig løsning \\
   
    	
\label{version_Systemark}
\end{longtabu}