
\chapter{Indledning}
I ST3PRJ arbejdes med blodtryksmålere. Vi har valgt at udarbejde en blodtryksmåler til forsknings brug. Blodtryksmåleren skal kunne modtage en spænding fra en transducer  og nulpunktsjustere og kalibere efter ønske. Signalet skal vises i en graf på et display, hvor værdier for puls, systoliske- og diastolisk tryk vises. Det er her fra at forskeren starter og gemmer målinger.\\
I Kravspecifikationen finde de krav som er blevet sat for systemet. Her under dem som blev stillet fra start, samt dem som vi har sat.\\
Under Systemarkitektur findes informationer om, hvordan software og hardware er opbygget.  I afsnittet integrationstest kan der læses om, hvordan vi har testet vores system.\\  

\subsubsection{Versionshistorik}

\begin{longtabu} to \linewidth{@{}l l l X[j]@{}}
    Version &    Dato &    Ansvarlig &    Beskrivelse\\[-1ex]
    \midrule
    1.0 &   04-11-2015	&   MHNK  &   Oprettelse af dokumenter \\
   
    	
\label{version_Systemark}
\end{longtabu}