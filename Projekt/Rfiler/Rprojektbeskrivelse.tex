\chapter{Projektbeskrivelse}
\section{Projektgennemførelse}
\section{Projektstyring}

\section{Metoder}
I tråd med ASE-modellen(beskrevet i Projektgennemførelse) er der blevet brugt accepttest, til at teste produktet.  
\section{Systemarkitektur}
\subsection{Hardware}
\subsection{Software}

\section{Problemidentifikation (design)}
\subsection{Hardware}
\subsection{Software}

\section{Implementering}
\section{GUI-beskrivelse}
\subsection{Algoritmer (grænseværdier)}
\subsection{Filteret/Ufiltreret}
\subsection{Lagring af data i Database}

\section{Test}
\section{Resultater og diskussion}
\section{Udviklingsværktøjer}
Gennem projektarbejdet har vi anvendt en række forskellige værktøjer til udvikling af blodtryksmåler-systemet. Disse er yderligere uddybet herunder.

\textbf{Visual Studio 2013}

Softwaredelen af projektets programmering er skrevet i sproget C-sharp. Her er Visual Studio 2013 anvendt som kompiler, da programmet gør det nemt at omskrive tekst til kode. Visual Studio 2013 indeholder også funktionen Windows Form Application, der visuelt kan fremstillede de ønskede resultater i form af knapper, grafer og labels mv. i en samlet brugergrænseflade, som aktøren interagerer med. 

\textbf{Microsoft Visio 2016}

Microsoft Visio er et tegne værktøj, der i dette projekt er anvendt til at designe både SysML og UML diagrammer, som benyttes ved organisering af hardware og software design. Microsoft Visio er det oplagte valg, da diagrammer lavet i programmet får et enkelt og overskueligt udseende, og dermed fremstår det tydeligt for læseren hvad diagrammet vil vise.

\textbf{Analog Discovery og Waveform fra Digilent}

Analog Discovery og waveform er i projektet benyttes som omformer og signal generator under testfasen. Her fungerer Analog Discovery som en waveform generator, så et analog signal kan sendes videre ind i lavpasfiltret, forstærkeren og derefter ind i DAQ’en. I den endelig implementering erstattes Analog Discovery og Waveform med transduceren. 

\textbf{NI-DAQmx}

NI-DAQmx er et værktøj udarbejdet af National Instruments, som anvendes til at omforme det indkomne analoge signal fra transduceren (Analog Discovery) til et digital signal. Værdier fra NI-DAQmx er af en type som kan anvendes i selve softwarekoden. 

\textbf{LaTeX}

LaTeX er anvendt i projektet til design og opsætning af projektrapport og projektdokumentation. LaTeX er god til tekstformatering, hvor opsætning og strukturer defineres samlet for hele en rapport, samt god til versionsstyring. Til at skrive selve koden benyttes programmet TeX-maker som kombiler. 

\section{Opnåede resultater}
\section{Perspektivering - Fremtidigt arbejde}
I fremtiden vil blodtryksmåleren kunne udviges gennem flere muligheder. Da blodtryksmåleren er lavet til forskningsbrug, er der ingen idé i at udvide mod patienter.  En forlængelse af systemet kunne derimod være en metode, som skal kunne vise gemte målinger. \newline Et log-in vindue er en anden ting som kunne forbedre systemet, for på den måde at skabe større sikkerhed for forskeren og dataen. Et log-in vindue vil gøre at, en forsker kan være sikker på at hans målinger og forskning ikke kan tilgås af andre. Det kræver en større udvidelse, hvor der skal laves et log-in vindue og en database, hvor password og brugernavn gemmes. Der skal også laves en metode, som kan tjekke om det indtastede password og brugernavn passer over ens med det i databasen. 
\newline
Generelt skal de standarter, som findes for blodtryksmålere undersøges grundigere. Specielt brugergrænsefladen, men også resten af systemet som enheder og visning af graf, skal rettes til efter de passende standarter.
\newline 
Fremtidsaspekter kunne også være, hvis systemet kunne tilpasses forskning mere.  Det kunne være gennem bedre navngivning af data eller et bedre overblik over, hvordan data bliver gemt, fx gennem en liste for de gemte målinger.   