\chapter{Projektbeskrivelse}
\section{Projektgennemførelse}
\section{Projektstyring}

\section{Metoder}
Til at kunne overskue arkitekture og designet af projektet, er flere forskellige arbejdsmetoder benyttet for at skabe det bedst mulige resultat. 
For at finde, hvad blodtryksmåleren skal gøre, er der blevet udarbejdet Use Cases. Disse beskriver systemet funktionalitet. Use Cases viser, hvad brugeren skal opleve fra systemet, men ikke, hvordan det sker. I Use Case diagrammet bliver det også vist, hvilke aktører der findes og hvordan de interagerer med systemet.   \\
I projektet bruges accepttest til at teste blodtryksmåleren. Dette gøres ud fra kravspecifikationerne, hvor det er angivet, hvilke krav der er til systemet. \\Accepttesten er en test, hvor der beskrives, hvad der skal ske og, hvad brugeren skal gøre. Testen er for at undersøge om produktet opfylder de krav som der er blevet sat for det. Accepttesten giver et godt overblik for udvikleren og for kunden, der nemt og hurtigt kan se om produktet virker som det skal. \\
\newline
Til  beskrivelse af design af software og hardware er diagrammer og skemaer blevet udarbejdet i SysML og UML. SysML er et grafisk modelleringssprog, som kan bruges til at overskueliggøre systemer. \\
Til software er der blandt andet lavet en applikationsmodel i SysML, som består af et domæne-, klasse- og sekvensdiagram. \\
Domænemodellen viser sammenhængen mellem blokkene i systemet. Blokkene findes i Use Casene og derved bliver disse to ting koblet sammen. \\
Klassedigrammet viser, hvilke metoder blokkene har og hvordan de kommunikerer med hinanden. Her findes domæne-, kontrol- og grænsefladeklasser. Kontrolklasserne beskriver, hvordan data behandles mellem domæne- og grænsefladeklasser. Domæneklasser indeholder funktionalitet fra den pågældende softwareblok. Grænsefladeklasserne viser, hvordan, systemet interagerer med omverdenen. Diagrammet gør det nemmere at fremme en lav kobling og høj samhørighed i softwaren.\\
Sekvensdiagrammet fortæller, hvad der sker i selve koden. Igen går det ud fra Use Casene, hvor vægten nu er på softwaredelen. Derved beskrives det, hvordan metoder bliver kaldt og hvordan de forskellige klasser interagerer. Hver Use Case skal her gennemgås i software, så der skabes et overblik over vejen gennem koden.\\
\newline 
For at skabe et overblik og indsigt i koden, er der i UML udarbejdet et aktivitetsdiagram og et klassediagram. Aktivitetsdiagrammet går i dybden med en specifik metode. Det er kun blevet gjort for relevante metoder. Her tydeliggøres det, hvordan hver metode fungere og, hvad den indeholder.  Klassediagrammet fortæller hvilke metoder, en klasse indeholder og hvordan klasserne hænger sammen.\\
\newline  
Til hardwaren er der blevet brugt Block Definition Diagram(BDD), som viser hvilke blokke et system indeholder og hvilke porte de har. BDD er lavet til at give et overblik over systemet. Ud fra BDD’et er et Internal Block Diagram(IDB) blevet lavet. Her vises, hvilke signaler som findes i systemet og hvordan de sendes rundt. Her vises portene igen og der skal være overensstemmelse  mellem BDD og IBD.    

\section{Systemarkitektur}
\subsection{Hardware}
\subsection{Software}

\section{Problemidentifikation (design)}
\subsection{Hardware}
\subsection{Software}

\section{Implementering}
\section{GUI-beskrivelse}
\subsection{Algoritmer (grænseværdier)}
\subsection{Filteret/Ufiltreret}
\subsection{Lagring af data i Database}

\section{Test}
\section{Resultater og diskussion}
\section{Udviklingsværktøjer}
Gennem projektarbejdet har vi anvendt en række forskellige værktøjer til udvikling af blodtryksmåler-systemet. Disse er yderligere uddybet herunder.

\textbf{Visual Studio 2013}

Softwaredelen af projektets programmering er skrevet i sproget C-sharp. Her er Visual Studio 2013 anvendt som kompiler, da programmet gør det nemt at omskrive tekst til kode. Visual Studio 2013 indeholder også funktionen Windows Form Application, der visuelt kan fremstillede de ønskede resultater i form af knapper, grafer og labels mv. i en samlet brugergrænseflade, som aktøren interagerer med. 

\textbf{Microsoft Visio 2016}

Microsoft Visio er et tegne værktøj, der i dette projekt er anvendt til at designe både SysML og UML diagrammer, som benyttes ved organisering af hardware og software design. Microsoft Visio er det oplagte valg, da diagrammer lavet i programmet får et enkelt og overskueligt udseende, og dermed fremstår det tydeligt for læseren hvad diagrammet vil vise.

\textbf{Analog Discovery og Waveform fra Digilent}

Analog Discovery og waveform er i projektet benyttes som omformer og signal generator under testfasen. Her fungerer Analog Discovery som en waveform generator, så et analog signal kan sendes videre ind i lavpasfiltret, forstærkeren og derefter ind i DAQ’en. I den endelig implementering erstattes Analog Discovery og Waveform med transduceren. 

\textbf{NI-DAQmx}

NI-DAQmx er et værktøj udarbejdet af National Instruments, som anvendes til at omforme det indkomne analoge signal fra transduceren (Analog Discovery) til et digital signal. Værdier fra NI-DAQmx er af en type som kan anvendes i selve softwarekoden. 

\textbf{LaTeX}

LaTeX er anvendt i projektet til design og opsætning af projektrapport og projektdokumentation. LaTeX er god til tekstformatering, hvor opsætning og strukturer defineres samlet for hele en rapport, samt god til versionsstyring. Til at skrive selve koden benyttes programmet TeX-maker som kombiler. 

\section{Opnåede resultater}
\section{Perspektivering - Fremtidigt arbejde}
I fremtiden vil blodtryksmåleren kunne udviges gennem flere muligheder. Da blodtryksmåleren er lavet til forskningsbrug, er der ingen idé i at udvide mod patienter.  En forlængelse af systemet kunne derimod være en metode, som skal kunne vise gemte målinger. \newline Et log-in vindue er en anden ting som kunne forbedre systemet, for på den måde at skabe større sikkerhed for forskeren og dataen. Et log-in vindue vil gøre at, en forsker kan være sikker på at hans målinger og forskning ikke kan tilgås af andre. Det kræver en større udvidelse, hvor der skal laves et log-in vindue og en database, hvor password og brugernavn gemmes. Der skal også laves en metode, som kan tjekke om det indtastede password og brugernavn passer over ens med det i databasen. 
\newline
Generelt skal de standarter, som findes for blodtryksmålere undersøges grundigere. Specielt brugergrænsefladen, men også resten af systemet som enheder og visning af graf, skal rettes til efter de passende standarter.
\newline 
Hvis systemet ydeligere skulle tilpasses forskning.  Det kunne være gennem en bedre navngivning af data i tabelle eller et bedre overblik over, hvordan data bliver gemt fx gennem en liste for de gemte målinger. På den måde vil det blive nemmere for forskeren at finde frem til gamle målinger.   
\newline 
I forhold til Hardware er målet, at det hele skal samles i en kasse. Så det på den måde ikke er muligt at ændre eller stille ved det. Derved skal filteret og forstærkningen laves på en printplade. Samtidigt skal det ved kassen være en plads til batterierne, hvor det er muligt at kunne skifte dem, når nødvendigt. Derved fås en kassen, som nemt kan flyttes rundt på og som ikke er i farer for at gå i stykker.   