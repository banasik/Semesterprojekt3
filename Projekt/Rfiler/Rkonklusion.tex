\chapter{Konklusion}
Formålet med projektet var, at udvikle en blodtryksmåler, som består af en hardware- og en softwaredel. Hardwaredelen består af en forstærker blok og et lavpasfilter. Hardwaren modtager et differentieret signal fra transduceren som forstærkes og filtreres inden det sendes gennem en DAQ og videre ind i softwaren. Softwaren vil gennem algoritmer, kunne detektere systoliske- og diastoliske værdier. Ud fra de to værdier ville programmet kunne udregne pulsen. \\
Da projektet blev lavet til forskningsbrug skulle forskeren kunne gemme sine målinger i en database. Herudover var formålet med projektet, at blodtryksmåleren skulle kunne blive kalibreret og nulpunktsjusteret, når det var ønsket. \\
I dette projekt er det lykkedes, at udvikle en repræsentativ prototype, der opfylder de overordnede formål. Systemet viser, på under tre sekunder, den systoliske- og diastoliske værdi. Det er derfor hurtigt og nemt, for en forsker, at få målbare data.\\
Selvom dette kun er en prototype, så er hardwaredelen blevet udviklet til at være brugervenlig. Kredsløbet er blevet loddet på et VEVO Board og lagt i en kasse. Batterierne holdes udenfor kassen, da det på denne måde vil være ubesværet for forskeren, at skifte batterierne. På kassen vil en lille diode lyse, når batterierne er sat til. \\
På trods af et funktionsdygtigt slutprodukt, må det konkluderes, at arbejdsprocessen blev stressede. Både hardware- og softwaredelen tog længere tid end forventet. Hensigten med den første tidsplan var god, men pga. DSB miniprojekter og en KSS eksamen, blev tidsplanen skubbet og især softwaredelen blev presset hen mod slutningen. Derfor er det ikke lykkedes at opfylde alle punkter i kravspecifikationen (disse punkter er beskrevet i problemrapporten). \\
Alt i alt er der blevet udviklet en fornuftig prototype, der med en smule videreudvikling, uden besvær, ville kunne bruges af en forsker. \\

