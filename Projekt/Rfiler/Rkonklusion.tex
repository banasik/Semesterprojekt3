\chapter{Konklusion}
Formålet med dette projekt var, at udvikle en hardwaredel, der skulle kunne trykændre et analogt signal. Gennem et analog filter skulle signalet forstærkes og sendes videre til en DAQ, der ville transformere det analoge signal til et digitalt. Herefter ville det digitale signal blive sendt ind i en softwaredel, som gennem algoritmer, ville kunne detektere systoliske- og diastoliske værdier. Ud fra de to værdier ville programmet kunne udregne pulsen. Herudover var formålet med projektet, at blodtryksmåleren skulle kunne blive kalibreret og nulpunktsjusteret af forskeren, når det var ønsket. Da projektet blev lavet til forskningsbrug skulle forskeren også kunne gemme sine målinger i en database.\\
I dette projekt er det lykkedes, at udvikle en repræsentativ prototype, der opfylder formålet. Systemet viser, på under tre sekunder, den systoliske- og diastoliske værdi og det er derfor hurtigt og nemt, for en forsker, at få målbare data.\\
Selvom dette kun er en prototype, så er hardwaredelen blevet udviklet til at være brugervenlig. Kredsløbet er blevet loddet på en printplade, i stedet for løst på et fumlebræt og printet er derefter lagt i en kasse. Batterierne holdes udenfor kassen, da det på denne måde vil være ubesværligt for forskeren, at skifte batterierne.   \\
Det er ikke lykkedes at opfylde alle punkter i kravspecifikationen (disse punkter er beskrevet i problemrapporten). \\
På trods af et funktionsdygtigt slutprodukt, må det konkluderes, at arbejdsprocessen blev en smule stressede. Både hardware- og softwaredelen tog længere tid end forventet. Hensigten med den første tidsplan var god, men pga. DSB miniprojekter og en KSS eksamen, blev tidsplanen skubbet en smule og især softwaredelen blev presset hen mod slutningen. Alt i alt er der blevet udviklet en fornuftig prototype, der med en lille smule videreudvikling, uden besvær, ville kunne bruges af en forsker. \\

