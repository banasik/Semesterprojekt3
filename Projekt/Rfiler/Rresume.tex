\chapter{Resumé}
Dette projekt beskæftiger sig med blodtryksmåling, hvor der ud fra et blodtrykssignal kan bestemmes en systolisk- og diastolisk værdi, samt en pulsværdi. Formålet er at bygge en hardwaredel der kan trykændre et analogt signal. Via et filter skal signalet forstærkes og sendes videre igennem en Data Acquisition (DAQ), der kan lave det analoge signal om til et digitalt signal. Herefter skal signalet videre ind i softwaren. Her er formålet at få programmeret en algoritme, der kan detektere en puls samt en systolisk- og diastolisk værdi. Disse systole- og diastoleværdier kan findes ud fra det digitale signal. Signalet bliver observeret fra starttidspunktet, hvor værdien for trykket begynder at vokse, indtil at grafen når maksimum. Denne værdi fra start til maks. er den systoliske værdi. Efter maksimum begynder grafen for signalet at falde igen, da trykket bliver mindre, indtil at det når sit lavpunkt, hvor værdien igen begyndte at stige en smule. Dette lavpunkt er værdien for diastolen. Grafen er periodisk og ud fra den systoliske- og diastoliske værdi kan pulsen findes. \\
Herudover skal programmet kunne kalibreres og nulpunktsjusteres når forskeren synes nødvendigt. Projektet er lavet til forskningsbrug og derfor blev det vurderet, at det er nødvendigt for en forsker, at kunne gemme sine målinger. Programmeringen skal derfor også gøre det muligt for forskeren, at kunne gemme sine målinger i en database.  \\
Resultatet for dette projekt viser at det er muligt vha. hardware- og softwaredelen, at måle et blodtryk. Gennem videreudvikling kan det blive muligt at detektere hypertension (Forhøjet blodtryk) og hypotension (Lavt blodtryk) i samme program. Dette ville være nyttigt, hvis projektet i fremtiden skulle bruges til patienter. I projektet er der dog udelukkende blevet arbejdet med måling af blodtryk.
