\chapter{Resumé}
Dette projekt beskæftiger sig med blodtryksmåling, hvor der ud fra et blodtrykssignal kan bestemmes en systolisk og diastolisk værdi, samt en puls værdi. Formålet er at bygge en hardwaredel, der kan bearbejde et analogt signal. Signalet skal forstærkes, filtreres og sendes igennem en Data Acquisition, der kan lave det analoge signal om til et digitalt signal. Herefter skal signalet ind i softwaredelen. Her er formålet at få programmeret algoritmer, der kan detektere puls, samt detekterer systolisk og diastolisk værdi.  Signalet bliver observeret fra starttidspunktet, hvor softwaren finder maksimums- og minimumsværdier for hvert tredje sekund. Maksimumsværdien svarer til den systoliske værdi, minimumsværdien svarer til den diastoliske værdi.\\
Herudover skal programmet kunne nulpunktsjusteres ved opstart og kalibreres når en forsker finder det nødvendigt. Projektet er lavet til forskningsbrug og derfor blev det vurderet, at det er nødvendigt for en forsker, at kunne gemme sine målinger. Programmeringen gør det derfor muligt for forskeren, at kunne gemme sine målinger i en database.  \\
Resultatet for dette projekt viser, at det er muligt vha. en hardware- og softwaredel, at måle et blodtryk. Gennem videreudvikling kan det blive muligt at detektere forhøjet blodtryk og lavt blodtryk i samme program. Dette ville være nyttigt, hvis projektet i fremtiden skulle bruges til patienter. I projektet er der dog udelukkende blevet arbejdet med måling af blodtryk til forskningsbrug.
