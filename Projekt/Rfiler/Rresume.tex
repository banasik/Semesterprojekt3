\chapter{Resumé}
Dette projekt har beskæftiget sig med blodtryksmåling, hvor der ud fra et blodtrykssignal kan bestemmes den systoliske- og diastoliske- og puls værdi. Formålet var, at bygge en hardwaredel, der kunne trykændre et analogt signal. Via et filter skulle signalet forstærkes og sende det videre igennem en Data acquisition (DAQ), der kunne lave det analoge signal om til et digitalt signal. Herefter skulle signalet løbe ind i softwaren. Her var formålet, at få programmeret en algoritme, der kunne detektere en puls, systolisk- og diastolisk værdi. Værdierne systolisk og diastolisk kunne findes ved, at kigge på værdien for tiden 0 sek. og frem, indtil den højeste værdi var nået og værdien igen ville falde. Den højeste værdi ville være den systoliske værdi. Når værdien så nåede den laveste, indtil den igen ville begynde at stige, vil denne laveste værdi være den diastoliske værdi. Ud fra den systoliske- og diastoliske værdi kunne pulsen findes. \\
Herudover skulle programmet kunne kalibreres og nulpunktsjusteres, når forsker synes nødvendigt. Projektet er lavet til forskningsbrug og derfor blev det vurderet, at det er nødvendigt for en forsker, at kunne gemme sine målinger. Programmeringen skulle derfor også gøre det muligt, for forskeren, at kunne gemme sine målinger i en database.  \\
Resultatet for dette projekt, viser at det er muligt vha. hardware- og softwaredelen, at måle et blodtryk. Gennem videreudvikling kunne det blive muligt at detektere hypotension og hypertension i samme program. Dette ville være nyttigt, hvis projektet i fremtiden skulle bruges til patienter. I projektet er der dog udelukkende blevet arbejdet med måling af blodtryk. 
