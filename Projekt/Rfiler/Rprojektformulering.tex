\chapter{Projektformulering og afgrænsning}
I daglig klinisk praksis er der ofte behov for kontinuert at monitorere patienters blodtryk, i særdeleshed på intensive afdelinger samt operationsstuer, hvor blodtrykket er en vigtig parameter i monitorering af deres helbredstilstand.

Denne kontinuere monitorering er også nødvendig i forskningsverdenen. Det er i forskerens interesse at kunne måle blodtrykket på et objekt, med f.eks. en mekanisk hjerteklap. Her skal det være muligt for forskeren at kunne aflæse diastole, systole og puls samt en pæn kurve over blodtrykket. Det er vores mål at opbygge et produkt, der kan tage de målinger der kommer fra tryktransduceren i volt og forstærke samt filtrere dem med fysiske komponente. Dette signal skal derefter konverteres fra analogt til digitalt. Herfra skal vi programmere en brugergrænseflade, der fremfører disse målinger samt gør det muligt for forskeren at gemme målingen i en database, til senere brug. Resultatet bliver et elektronisk kredsløb med forbindelse til et software program.
For at de gemte data kan sammenlignes, kræver det at de alle er blevet gemt med samme forudsætning. Dette bliver ordnet i programmeringen med beregninger. Når forskeren kigger på blodtryksgrafen, er det hans interesse at vælge om målingen skal være filtreret eller ej. Dette er implementeret i programmering. 
\subsubsection{Ansvarsområder}

Idet gruppens størrelse ikke lægger op til samlet, at arbejde på alle dele på samme tid, er projektets ansvarsområder blevet fordelt som følgende:

\begin{longtabu} to \linewidth{@{}l l l X[j]@{}}
    Navn &    Ansvarsområder &    \\[-1ex]
    \midrule
    Ditte Heebøll Callesen &   Hardwaredesign, dokumentation	&    \\
    Albert Jakob Fredshavn &   Hardwaredesign, dokumentation	&    \\
    Martin Banasik         &   Hardwaredesign, dokumentation	&    \\
    Johan Mathias Munk     &   Softwaredesign, algoritmeopbygning, dokumentation &    \\
    Mette Hammer Nielsen-Kudsk  &   Softwaredesign, algoritmeopbygning, dokumentation	&    \\
   	Anne Hoelgaard    &   Softwaredesign, algoritmeopbygning, dokumentation	&    \\
\label{version_Systemark}
\end{longtabu}