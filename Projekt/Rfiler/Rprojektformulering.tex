\chapter{Projektformulering og afgrænsning}
I daglig klinisk praksis er der ofte behov for kontinuert at monitorere patienters blodtryk, i særdeleshed på intensive afdelinger, samt operationsstuer, hvor blodtrykket er en vigtig parameter i monitorering af patienters kardiovaskulære status.\\
Denne kontinuerlige monitorering er også nødvendig i forskningsverdenen. Det er i forskerens interesse, at kunne måle blodtrykket når der laves hæmodynamiske undersøgelser. Her skal det være muligt for forskeren at kunne aflæse det diastoliske tryk, systoliske tryk og pulsen, samt få vist en pæn kurve over blodtrykket. Det er målet, at opbygge en prototype, der kan registrere de spændinger, i millivolt (mV), der kommer fra transduceren og analogt forstærke samt filtrere signalet. Dette signal skal derefter konverteres til det digitale domæne. \\
Herfra skal der programmeres en brugergrænseflade, der fremfører disse målinger, samt gør det muligt, for forskeren, at gemme målingen i en database til senere brug. Resultatet bliver derfor et elektronisk kredsløb med forbindelse til et softwareprogram.
For at det gemte data kan sammenlignes, kræver det at alt er blevet gemt med samme forudsætning, dvs. at alt data er gemt som rådata. Dette bliver håndteret i softwaredelen, hvor beregninger implementeres. Når forskeren kigger på blodtryksgrafen, vil det filtreret signal vises. I tilfælde af at det er i forskers interesse at se på et ufiltreret signal, vil dette være muligt, ved et tryk på en knap.

\textbf{MoSCoW }

\textit{Must}
\begin{itemize}
\item Et elektronisk kredsløb, som forstærker signalet fra transduceren og filtrerer det med ét indbygget analogt filter
\item Et program til at vise blodtrykket som funktion af tiden. Programmet skal opfylde en række obligatoriske krav. Det skal kunne:
\begin{itemize}
\item Programmeres i C\#
\item Kunne kalibrere blodtrykssignalet og foretage en nulpunktsjustering
\item Vise blodtrykssignalet kontinuert
\item Kunne gemme de målte data i en database
\item Kunne filtrere blodtrykket i selve programmet via et digitalt filter, som skal kunne slås til og fra (monitor mode = filtreret og afrundet; signaldiagnose mode = råt signal med alle udsving, ufiltreret)
\item Afbildning af systolisk/diastolisk blodtryk med tal
\end{itemize}
\end{itemize}
\textit{Could}
\begin{itemize}
\item Hardwaredelen skal bestå af ét Veroboard med påsatte komponenter

\end{itemize} 
\textit{Would}
\begin{itemize}
\item Alarmering, hvis blodtrykket afviger fra indbyggede grænseværdier
\item Forskeren skal kunne hente de gemte data ned igen
\end{itemize}

\subsubsection{Ansvarsområder}

Idet gruppens størrelse ikke lægger op til samlet at arbejde på alle dele samtidig, er projektets ansvarsområder blevet fordelt som følgende:

\begin{longtabu} to \linewidth{@{}l l l X[j]@{}}
    \textbf{Navn} &   \textbf{ Ansvarsområder} &    \\[-1ex]
    \midrule
    Ditte Heebøll Callesen &   Hardware, dokumentation, rapport	&    \\
    Albert Jakob Fredshavn &   Hardware, dokumentation, rapport	&    \\
    Martin Banasik         &   Hardware, dokumentation, rapport	&    \\
    Johan Mathias Munk     &   Software, dokumentation, rapport &    \\
    Mette Hammer Nielsen-Kudsk &   Software, dokumentation, rapport &    \\
   	Anne Hoelgaard    &   Software, dokumentation, rapport	&    \\
\label{version_Systemark}
%\end{longtabu}

%\begin{longtabu} to \linewidth{@{}l l l X[j]@{}}
    \textbf{Navn} &    \textbf{Skrevet afsnit i Rapport} &\\[-1ex]
    \midrule
    Ditte Heebøll Callesen & Metode, systemarkitektur (hardware) & \\
     & og perspektivering & \\
    Albert Jakob Fredshavn &  Projektgennemførelse og projektstyring & \\
    Martin Banasik         & Systemarkitektur (hardware) og opnåede resultater & \\
    Johan Mathias Munk     & Projektformulering, afgrænsning &  \\ & og systemarkitektur (software) & \\
    Mette Hammer Nielsen-Kudsk & Resumé, abstract, indledning, blodtryk, resultater, diskussion &  \\
     & og konklusion &  \\
   	Anne Hoelgaard   	 	& Systembeskrivelse, Krav, Systemarkitektur (software) & \\ & og udviklingsværktøjer & \\
%\label{version_Systemark}
\end{longtabu}

