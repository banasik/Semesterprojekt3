\chapter{Faglig viden om blodtryk}
\section{Hvad betyder blodtryk?} 
Alle har på et tidspunkt i deres liv fået målt blodtryk, men hvad betyder det egentlig?
Blodtryksmåling er en meget enkelt undersøgelse, der giver vigtige informationer omkring blodkars og hjertets tilstand. 
Blodtryk måles i en enhed, der hedder millimeter kviksølv (mmHg). Normalt ligger blodtrykket omkring 120/80 mmHg. 
Det "høje blodtryk" er det systoliske tryk, der måles når venstre ventrikel trækker sig sammen og iltet blod pumpes rundt i kroppen.  
Det "lave blodtryk" er det diastoliske tryk, der måles når hjertet slapper af mellem 2 slag. 
\textbf{Beskriv systole og diastole lidt mere!!!} 
\textit{Her er mere om de diastoliske tryk: Når hjertet slapper af, bliver det venstre hjertekammer fyldt med blod, der er kommet retur fra lungerne. Herefter trækker det venstre hjertekammer sig sammen, og pumper blodet ud i arterierne (hovedpulsårerne). Blodtrykket i hovedpulsårerne er højere når kammeret trækker sig sammen, eftersom blodet bliver 'presset' ud i årerne. Trykket er lavere når hjertet slapper af (det diastoliske tryk), eftersom der ikke bliver presset blod ud i årerne.!!! }
Noget med at hjertet pumper ilt rundt til hele kroppen (muskler og organer). 
Når de røde blodlegemer har afgivet ilt til de forskellige væv, løber blodet gennem tilbageløbs-blodårerne (venerne) tilbage til hjerte
   

\section{Forhøjet blodtryk} 
Det er vigtigt, ofte at få målt sit blodtryk, da forhøjet blodtryk ikke kan mærkes og er den vigtigste årsag til hjerte-kredsløbs sygdomme. Hypertension kan føre til mange medicinske tilstande, såsom åreforkalkning, hjertesvigt (nedsat pumpefunktion, som medfører væskeophobning i kroppen), hjerteinfarkt, slagtilfælde, hjertekrampe og nyreskade, for bare at nævne de vigtigste. 
Der er tale om hypertension når blodtrykket viser 140/90 mmHg eller højere.  
Forhøjet blodtryk behandles med lægemidler (medicin) - blodtryksnedsættende medicin. 
når blodet bliver presset igennem pulsårerne med et højere tryk end normalt.

Risikofaktorer:
Alder. Arv. Overvægt og fedme. Mangle på motion. Rygning. For meget salt. Overforbrug af alkohol. Stress. 

\section{Lavt blodtryk} 
 Hvis man har diabetes eller høj risiko for hjerte- og kredsløbssygdomme , bør blodtrykket være endnu lavere. Det kan ses ved f.eks. traumer, hjertesvigt, Addisons sygdom  eller diabetes.
Ved en række medicinske tilstande optræder lavt blodtryk, ved for eksempel hjertesygdomme, lavt stofskifte, nedsat funktion af binyrer, leversygdomme m.v.
 
Hypotension - 90/60 mmHg. Kun ét af tallene skal være lavt for at dit blodtryk kan regnes for lavere end normalt. Med andre ord kan du sagtens have et perfekt systolisk blodtryk på 115, men hvis dit diastoliske blodtryk samtidigt er 50, regnes dit blodtryk for lavere end normalt.

Folk med lavt blodtryk har heldigvis mindre risiko for at få hjerteanfald, nyresygdomme og hjertesygdomme - sammenlignet med personer, der lider af forhøjet blodtryk.

Blodtrykket er først for lavt når tilstanden fører til symptomer eller tegn på, at der ikke flyder tilstrækkeligt med blod gennem blodårerne. Når blodforsyningen til kroppen vigtige organer (herunder hjernen, hjertet og nyrerne) bliver begrænset, risikerer man at der ikke bliver leveret tilstrækkeligt med ilt til organerne. Dette medfører at organerne ikke fungerer optimalt samt at de eventuelt kan blive midlertidigt eller permanent skadet af tilstanden. (træthed, svimmelhed, besvimelse, koncentrationsbesvær, kvalme, depression).  

Risikofaktorer: 
Alder. Medicinforbrug. Visse sygdomme.

http://prodoktor.dk/forhøjet-blodtryk/
http://prodoktor.dk/lavt-blodtryk/ 


