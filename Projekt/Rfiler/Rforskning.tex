\chapter{Blodtryk}
\section{Hvad betyder blodtryk?} 
Alle har på et tidspunkt i deres liv fået målt blodtryk, men hvad betyder det egentlig?
Blodtryksmåling er en meget enkelt undersøgelse, der giver vigtige informationer om blodkarrenes og hjertets tilstand. Blodtryk kan måles både invasivt og ikke-invasivt. Dette projekt omhandler invasivt blodtryksmåling, som er en måling af trykket direkte i en blodåre. 
Blodtrykket måles i enheden millimeter kviksølv (mmHg). Et normalt blodtryk ligger på omkring 120/80 mmHg \cite{Blodtryk}. \\ 
Hjertet pumper iltet blod ud i hele kroppen via arteriesystemet og sørger dermed for tilførslen af ilt og næringssubstanser til alle muskler og organer. De røde blodlegemer (erythrocytterne) er en vigtig bestanddel af blodet. Det er hæmoglobinet i de røde blodlegemer, som binder ilten og sørger for at ilten transporteres frem til vævene i kroppen. Når ilten er afgivet fra blodet til musklerne og andre væv, transporteres blodet tilbage til højre side af hjertet via venesystemet. Det af-iltede blod pumpes af højre hjertepumpekammer ud i lungerne, hvor blodet iltes på ny og derfra strømmer til venstre hjertehalvdel for igen at blive pumpet ud i kroppen af venstre hjertepumpekammer. \\
Det høje tryk er det systoliske tryk, som kan måles når venstre ventrikel trækker sig sammen. Det diastoliske blodtryk er blodtrykket i hjertets afslapningsfase (diastolen). Når hjertet trækker sig sammen (systolen) skaber det en trykbølge som forplanter sig ud igennem arteriesystemet. Trykbølgen kan erkendes som pulsen, der let kan mærkes f.eks. ved palpation af a. radialis ved håndleddet. Trykket i venesystemet er meget lavere end i arteriesystemet, da blodet passivt blot skal strømme tilbage til højre forkammer, hvor trykket er lavt. \\
Venstre ventrikel pumper iltet blod, under højt tryk, ud i aorta og arterierne. Disse blodårer er derfor tykvæggede og elastiske i modsætning til venerne, der er ganske tyndvæggede, fordi de kun udsættes for et lavt tryk.. Hjertet overfører, gennem systolen, energi til arterievæggen, som bruges i den resterende del af hjertets cyklus, til at presse blod gennem karsystemet. 


For at forstå det følgende afsnit introduceres tre vigtige begreber nedenfor: 
\begin{itemize}
\item Væskevolumen, der løber igennem et rør pr. tidsenhed, kaldes for væskestrømmen.
\item Distancen, som en væske flytter sig pr. tidsenhed er strømningshastigheden.
\item Blodvolumen, der løber gennem et væv pr. tids- og vægtenhed er gennemblødning.
\end{itemize}

Der er en trykforskel imellem begyndelsen og slutningen af et rør. Væskestrømmen i røret afhænger af trykforskellen hen over røret og modstanden i røret. Dette kan udtrykkes i følgende ligning som minder om Ohms lov for elektriske kredsløb:  
\begin{equation}
V$æ$skestr$ø$m(Q)= \frac{Trykforskel(\Delta P)}{Modstand(R)}
\end{equation}
Drivkraften for væskestrømningen (Q) er trykforskellen $(\Delta P)$ gennem røret. Hjertets kontraktioner gør, at strømmen i røret går fra et højere, til et lavere tryk. 
Modstanden (R) i en arterie er bestemt af bl.a. gnidningsmodstanden mellem arterievæggen og blodet, blodets viskositet og diameteren af arterien. Når blodet løber igennem arterierne, falder trykket efterhånden i blodet. Ved stigende modstand mod væskestrømmen forøges trykfaldet. 
Når modstanden i rørvæggen stiger, formindskes væskestrømningen, hvis trykforskellen samtidig ikke er steget.

Et rørs modstand bestemmes ud fra tre parametre: 

\begin{itemize}
\item Længden af røret.
\item Den indre diameter på røret. 
\item Viskositeten af væsken.
\end{itemize}

Jo kraftigere hjertet pumper, desto større bliver trykforskellen og dermed blodstrømningen. 
Blodkarrets diameter er det, der har størst betydning for modstanden mod blodstrømmen. Hvis blodet presses igennem et snævert kar, er der en større del af blodet, der er tæt på karvæggen og bliver derved bremset af friktionskraft. Modsat, hvis diameteren på karret havde været større, ville en mindre del af blodet være i kontakt med væggen og derved ville det ikke blive bremset lige så meget. Modstanden er derfor mindre og blodstrømningen større, i et stort kar. \\
Blodets viskositet stiger, jo flere røde blodlegemer, der findes i blodet. Jo flere røde blodlegemer, desto højere viskositet. Hvis blodet har en høj viskositet og derved er tyktflydende, så skal der et større tryk til at holde en hvis væskestrøm \citep{Blodtryk}. 
   
\section{Hypertension} 
Hypertension er en meget almindelig lidelse, ca. 30\% af den danske befolkning har forhøjet arterielt blodtryk \cite{Hypertension}. Derfor er det vigtigt ofte at få målt sit blodtryk, da forhøjet blodtryk ikke kan mærkes og er den vigtigste årsag til hjerte-kar-sygdomme. 
Der er tale om hypertension når blodtrykket er 140/90 mmHg eller højere.
Ved hypertension bliver arbejdsbelastningen af hjertet forøget, da der skal pumpes blod ud af hjertet mod en større modstand i arteriesystemet. \\
Forhøjet blodtryk gør at arbejdsbelastningen bliver større. Derved sker der lige så stille en fortykkelse af muskulaturen i venstre ventrikel, da hjertet skal bruge flere kræfter på at pumpe blodet ud i aorta. 
Det øgede tryk påvirker blodkarrenes belastning, og kan medføre at mindre blodkar kan briste under det store tryk. Hvis der er tale om blodkar i hjernen kan dette føre til en hjerneblødning. Hypertension er den hyppigste årsag til hjerneblødninger.
Hypertension kan føre til en række andre komplikationer i form af åreforkalkning, hjerteinsufficiens \cite{Hjerteinsufficiens}, akut myokardieinfarkt, hjertekrampe, nyreskader og apopleksi.
Forhøjet blodtryk behandles med lægemidler, i form af blodtryksnedsættende medicin (antihypertensiva). Samtidig er non-farmakologiske metoder, som rygestop, motion, reduktion af saltindtagelse, vægttab og reduktion af alkoholforbrug vigtige i behandlingen af hypertension. 

\section{Hypotension} 
Hypotension er modsat hypertension et lavt blodtryk, og ikke nær så almindeligt. Der er tale om hypotension når blodtrykket er 90/60 mmHg. Hypotension ses især ved en række alvorlige akutte tilstande som akut myokardieinfarkt, lungeemboli, sepsis eller alvorlige blødninger. Patienten kan i disse situationer være i en livstruende shock tilstand. Lavt blodtryk kan også i sjældne tilfælde forekomme kronisk, især ved sjældne stofskiftesygdomme med nedsat produktion af binyrebarkhormoner \cite{Hypotension}.   