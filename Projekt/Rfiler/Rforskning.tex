\chapter{Faglig viden om blodtryk}
\section{Hvad betyder blodtryk?} 
Alle har på et tidspunkt i deres liv fået målt blodtryk, men hvad betyder det egentlig?
Blodtryksmåling er en meget enkelt undersøgelse, der giver vigtige informationer omkring blodkars og hjertets tilstand. Blodtryk kan måles både ikke invasivt og invasivt. Dette projekt omhandler invasivt blodtryksmåling. 
Blodtrykket måles i enheden millimeter kviksølv (mmHg). Et normalt blodtryk ligger omkring 120/80 mmHg. 
Hjertet pumper iltet blod ud i hele kroppen og giver energi til alle muskler og organer. De røde blodlegemer, i blodet, er dem, der afgiver ilt til vores væv. Når ilten er blevet afgivet til musklerne og organerne, skal blodet tilbage til hjertet, igennem venerne.  
Det høje tryk er det systoliske tryk, som kan måles når den venstre ventrikel trækker sig sammen og iltet blod pumpes rundt i kroppen. 
Når blodet kommer retur fra lungerne, løber det, når hjertet slapper af, ind i det venstre hjertekammer. Det venstre hjertekammer trækkes herefter sammen og pumper blodet ud i arterierne.Blodtrykket er herved højere i arterierne, når det venstre hjertekammer trækker sig sammen, fordi blodet bliver presset ud i arene. Trykket er derved lavere når hjertet slapper af, da der ikke bliver presset blod ud i årerne. 
Det lave tryk er det diastoliske tryk, der kan måles når hjertet slapper af mellem 2 hjerteslag. 
Venstre ventrikel pumper iltet blod ud i aorta og arterierne. Disse er derfor, blodkarrene i kroppen, der udsættes for det største tryk, hvilket er derfor at de stærke og elastiske. 
Hjertet overfører, gennem systolen, energi til arterievæggen, som bruges i den resterende del af hjertets cyklus, til at presse blod gennem karsystemet. 

For at forstå det følgende afsnit er der nogle begreber, der skal på plads. 
\begin{itemize}
\item Væskevolumen, der løbet igennem et rær pr. tidsenhed, kaldes for væskestrømmen.
\item Distancen, som en væske flytter sig pr. tidsenhed er strømningshastigheden.
\item Blodvolumen, der løber gennem et væv pr. tids- og vægtenhed er gennemblødning.
\end{itemize}

Der er en trykforskel i begyndelsen og slutningen af et rør. I begyndelsen af røret stiger væskestrømningen med trykforskellen og i slutningen af røret aftages væskestrømningen med rørets modstand (R). 

Væskestrømmens definition: 
$$ Væskestrøm(Q)= \frac{Trykforskel(\Delta P)}{Modstand(R)}$$

Drivkraften for væskestrømningen er trykforskellen $(\Delta P)$ gennem røret. Hjertets kontraktioner gør, at strømmen i røret går fra et højere, til et lavere tryk. 
Modstanden, der er tale om, er gnidningsmodstanden mellem rørvæggen, der er rolig og blodet, der bevæger sig. 
Når blodet løber igennem arterierne, falder trykket efterhånden i blodet. Ved stigende modstand mod væskestrømmen, stiger trykfaldet. 
Når modstanden i rørvæggen derimod stiger, formindskes væskestrømningen, hvis trykforskellen ikke stiger.
Når modstanden i rørvæggen stiger, så skal væskestrømningen ikke ændres, derfor skal trykforskellen stige, hvilket netop gør at hjertet bliver nød til at arbejde hårdere. 

Et rørs modstand bestemmes ud fra tre parametre: 

\begin{itemize}
\item Længden af røret.
\item Den indre diameter på røret. 
\item Viskositeten af væsken.
\end{itemize}

Jo kraftigere hjertet pumper, desto større bliver trykforskellen og dermed blodstrømningen. 
Blodkarrets diameter, er det, der har størst betydning for modstanden mod blodstrømmen. Hvis blodet presses igennem et snævert kar, så er der en større del af blodet, der er tæt på karvæggen og bremses derved af friktionskræfterne. Hvorimod, hvis diameteren på karret havde været større, så ville en mindre del af blodet være i kontakt med væggen og derved bremses der ikke ligeså meget. Modstanden er derfor mindre og blodstrømningen større i et stort kar. 
Blodets viskositet stiger jo flere røde blodlegemer, der findes i blodet. Jo flere røde blodlegemer, desto højere viskositet. Hvis blodet har en høj viskositet og derved er tyktflydende, så skal der et større tryk til at holde en vis væskestrøm. 


   
\section{Hypertension} 
Hypertension er en meget almindelig lidelse, ca. 30\% af den danske befolkning har forhøjet arterielt blodtryk. Derfor er det vigtigt, ofte at få målt sit blodtryk, da forhøjet blodtryk ikke kan mærkes og er den vigtigste årsag til hjerte-kar-sygdomme. 
Der er tale om hypertension når blodtrykket viser 140/90 mmHg eller højere.
Ved hypertension bliver arbejdsbelastningen af hjertet, forøget, da der skal pumpes blod ud af hjertet med en større modstand. Når blodet bliver presset igennem pulsårerne med et højere tryk end normalt. 
Forhøjet blodtryk gør at arbejdsbelastningen bliver større. Derved sker der ligeså stille en fortykkelse af muskulaturen i den venstre ventrikel, der skal bruge flere kræfter på at pumpe blodet ud i aorta. 
De fysiske påvirkninger på blodkarrene bliver også øget. Små blodkar kan derfor hurtigere briste. Det kan specielt være alvorligt, hvis det er hjerneblodkar, der brister da man så har fået en hjerneblødning. 
Hypertension kan føre til og er resultat af mange hjerte-kar-sygdomme såsom åreforkalkning, hjerteinsufficiens, akut myokardieinfarkt, hjertekrampe, nyreskade og apopleksi, for blot at nævne de vigtigste. 
Forhøjet blodtryk behandles med lægemidler - blodtryksnedsættende medicin. Blodtrykket skal ca. reduceres med 20/10 mmHg. Non-farmakologiske metoder er rygestop, motion, reduktion af saltindtagelse, vægttab og reduktion af alkoholforbrug. 

\section{Hypotension} 
Hypotension er ikke en ligeså almindelig lidelse som hypertension. Folk med lavt blodtryk har heldigvis mindre risiko for at få hjerteanfald, nyresygdomme og hjertesygdomme - sammenlignet med personer, der lider af forhøjet blodtryk.
Der er tale om hypotension når blodtrykket viser 90/60 mmHg. Kun en af disse værdier behøver at være for lavt for at blodtrykket defineres som, lavere end normalt. Man kan altså godt have et systolisk tryk på 120 mmHg, men hvis det diastoliske tryk så er 50 mmHg, så er blodtrykket for lavt.  



Blodtrykket er først for lavt når tilstanden fører til symptomer eller tegn på, at der ikke flyder tilstrækkeligt med blod gennem blodårerne. Når blodforsyningen til kroppens vigtige organer (herunder hjernen, hjertet og nyrerne) bliver begrænset, risikerer man at der ikke bliver leveret tilstrækkeligt med ilt til organerne. Dette medfører at organerne ikke fungerer optimalt samt at de eventuelt kan blive midlertidigt eller permanent skadet af tilstanden.

Hvis man har diabetes eller høj risiko for hjerte-kar-sygdomme, bør blodtrykket være endnu lavere. Det kan ses ved f.eks. traumer, hjertesvigt, Addisons sygdom  eller diabetes.
Ved en række medicinske tilstande optræder lavt blodtryk, f.eks. ved hjertesygdomme, lavt stofskifte, nedsat funktion af binyrer, leversygdomme m.v.

 (træthed, svimmelhed, besvimelse, koncentrationsbesvær, kvalme, depression).  

Risikofaktorer: 
Alder. Medicinforbrug. Visse sygdomme.

http://prodoktor.dk/forhøjet-blodtryk/
http://prodoktor.dk/lavt-blodtryk/ 


