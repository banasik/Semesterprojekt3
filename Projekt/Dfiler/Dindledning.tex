\chapter{Indledning}
I  dag bruges blodtryksmålere mange steder, både på hospitalet og i hjemmet. Blodtryksmålere kan måle en persons blodtryk, hvor den viser puls, samt diastoliske- og systoliske tryk i numerisk form og afbilledet i en graf.\\ 
Vi har valgt at arbejde ud fra, at blodtryksmåleren skal bruges til forskning. Derfor skal systemet gemme samtlige målinger, så en forsker senere kan tilgå dem. Samtidig skal puls og tryk vises på en graf, som skal være nem at aflæse. Brugeren vil kunne benytte målere gennem et interface, hvor han kan starte og gemme målinger. Det er også her grafen vises. \\
Der var fra start givet en række krav til systemet, samtidig har gruppen valgt at tilføje nogle flere for at få de ting løst, gruppen synes var vigtigt. Disse kan findes i Kravspecifikationen og under Krav.\\
Nærmere informationer om opbygning af hardware og software kan findes under Systemarkitektur, som er delt ind efter Hardware og Software. Her under findes også Modultest. 
Under Modultest kan det læses, hvordan vi har testet systemet samlet og enkelt hvis for hardware og software. Under Accepttest ses det, om systemet opfylder kravene der blev sat.\\ 


\textbf{Ansvarsområde} \\
\textbf{Initialer: } \\
Albert Jakob Fredshavn - AJF \\
Martin Banasik - MBA \\
Mette Hammer Nielsen-Kudsk - MHNK \\
Ditte Heebøll Callesen - DHC \\
Johan Mathias Munk - JMM \\
Anne Bundgaard Hoelgaard - ABH \\


\begin{longtabu} to \linewidth{@{}  l X[j]@{}}
    Afsnit &    Ansvarlig\\[-1ex]
    \midrule
    Indledning & \\
    Kravspecifikation & \\
    Hardware arkitektur & \\
    Software arkitektur & \\
    Software implementering & \\
    Accepttest & \\
    Fejlrapport & \\
    
    
    

\end{longtabu}