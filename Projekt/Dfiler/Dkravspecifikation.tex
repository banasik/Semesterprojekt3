\chapter{Kravspecifikation}

\begin{longtabu} to \linewidth{@{}l l l X[j]@{}}
    Version &    Dato &    Ansvarlig &    Beskrivelse\\[-1ex]
    \midrule
    1.0 &    21-09-2015 &    XXX &    Påbegyndt tilrettelse af UC1 og UC2
    
\label{version_Systemark}
\end{longtabu}


\section{Indledning}


\section{Funktionelle krav}
 

\subsection{Aktør-kontekstdiagram}

\begin{figure}[htb]
	\centering
	%\includegraphics[width=1\textwidth]{Figurer/Snip20150409_30}
	\caption{Aktør-kontekstdiagram}
	\label{fig:aktoerbeskrivelse}
\end{figure}



\subsection{Aktørbeskrivelse}

\begin{table}[H]
\begin{tabularx}{\textwidth}{l l X}
     Aktørnavn  & Type      & Beskrivelse \\ \midrule
     Bruger   & Primær    & Brugeren er den aktør, der ønsker at foretage målingerne, som omfatter EKG samt diagnosticering af artieflimmer. Brugeren er en person, der har kendskab til EKG-systemet. Fx sundhedsfaglig personale \\ 						  									  \addlinespace[2mm]
     
   
     \bottomrule                                                                                                                   
    \end{tabularx}
    \caption {Aktørbeskrivelse}
    \label{tab:aktoerbeskrivelse}
	
\end{table}

\subsection{Use case-diagram}

\begin{figure}[H]
	\centering
	%\includegraphics[width=1\textwidth]{Figurer/Snip20150518_11}
	\caption{Use case-diagram}
	\label{fig:Use Cases}
\end{figure}



\subsection{Use Cases}

\begin{longtabu} to \linewidth{@{}l r X[j]@{}} %UC1%
	{\large \textbf{Use Case 1}} && \\
	\toprule
	Scenarie && Hovedscenarie\\
	Navn && Foretag nulpunktsjustering\\
	Mål && At få foretaget en nulpunktsjustering\\
	Initiering && Startes af Forsker\\
	Aktører && Forsker (primær)\\
	Referencer && \\
	Samtidige forekomster  && En nulpunktsjustering pr. kørsel \\
	Forudsætninger && Alle systemer er ledige og operationel\\ 
	Resultat && Bliver foretaget nulpunktsjustering efter ønske\\ \midrule
	Hovedscenarie &    1. &		Popup vindue for nulpunktsjustering er åbent.\\				 	
	&    2. & Forsker trykker på:\\ 
	& &	a. Ja-knap for at starte en nulpunktsjustering.\\[-1ex]
	& &		 i. Nulpunktsjustering foretages og vinduet lukker efter endt justering.\\[-1ex]
	& &  b. Nej-knap for ikke at få foretaget en nulpunktsjustering.\\[-1ex]
	& &   ii. Vinduet lukker.\\[-1ex]	
	Undtagelser && -  \\ \bottomrule
	
	\caption{Fully dressed Use Case 1}
	\label{UC1}
\end{longtabu}

\begin{longtabu} to \linewidth{@{}l r X[j]@{}} %UC1%
	{\large \textbf{Use Case 2}} && \\
	\toprule
	Scenarie && Hovedscenarie\\
	Navn && Foretag kalibrering\\
	Mål && At få foretaget en kalibrering\\
	Initiering && Startes af Forsker\\
	Aktører && Forsker (primær)\\
	Referencer && Use Case 1\\
	Samtidige forekomster  && En kalibrering pr. kørsel \\
	Forudsætninger && Alle systemer er ledige og operationel\\ 
	Resultat && Bliver foretaget kalibrering efter ønske\\ \midrule
	Hovedscenarie &    1. &		Popup vindue for kalibrering er åbent.\\				 	
	&    2. & Forsker trykker på:\\ 
	& &	a. Ja-knap for at starte en kalibrering.\\[-1ex]
	& &		 i. Kalibrering foretages og vinduet lukker efter endt kalibrering.\\[-1ex]
	& &  b. Nej-knap for ikke at få foretaget en kalibrering.\\[-1ex]
	& &   ii. Vinduet lukker.\\[-1ex]	
	Undtagelser && -  \\ \bottomrule
	
	\caption{Fully dressed Use Case 2}
	\label{UC2}
\end{longtabu}

\begin{longtabu} to \linewidth{@{}l r X[j]@{}} %UC1%
    {\large \textbf{Use Case 3}} && \\
    \toprule
    Scenarie && Hovedscenarie\\
    Navn && Start Måling\\
    Mål && At få foretaget en blodtryksmåling\\
    Initiering && Startes af Forsker\\
    Aktører && Forsker (primær)\\
    Referencer && Use Case 1, Use Case 2\\
    Samtidige forekomster  && Et signal pr. måling\\
    Forudsætninger && Alle systemer kører og er klar til at foretage en måling\\ 
    Resultat && Systole, diastole og puls bliver vist på GUI\\ \midrule
    Hovedscenarie &    1. &		Forsker indtaster Forsøgsnummer\\[-1ex]
    &     & 	[\textit{Undtagelse 1}]:	Forsker vælger ufiltreret signal 	\\	 	
                &    2. & Forsker trykker på Start-knap på GUI\\[-1ex]
                &    3. & Filtreret signal for blodtryk vises på GUI\\[-1ex]
                &    4. & Systolisk og diastolisk blodtryk samt puls bliver vist i bokse på GUI\\[-1ex] \midrule
                
    Undtagelser && [\textit{Undtagelse 1}]	Forsker vælger ufiltreret signal.\\ 
    &	 & Use Case fortsættes fra punkt 2\\ 
    \bottomrule
\caption{Fully dressed Use Case 3.}
\label{UC3}
\end{longtabu}


\begin{longtabu} to \linewidth{@{}l r X[j]@{}} %UC2%
    {\large \textbf{Use Case 4}} && \\
    \toprule
    Scenarie && Hovedscenarie\\
    Navn && Gem data\\
    Mål && At gemme data\\
    Initiering && Forsker\\
    Aktører && Forsker (primær), Database(sekundær)\\
    Referencer && Use Case 1 - 3 \\
    Samtidige forekomster  &&  pr. måling\\
    Forudsætninger && Use Case 1  -3 er kørt succesfuldt\\
    Resultat && Systoliske og diastoliske blodtryk og puls er blevet gemt i en Database under Forsøgsnummer\\ \midrule
    Hovedscenarie &    1. &		Forsker trykker på Gem-knap\\[-1ex] 
    &	& [\textit{Undtagelse 1:}] Forsker trykker på Stop-knap\\
                  &    2. &     Systemet gemmer de fremadrettede 30 sekunders ufiltreret signal i Database\\
                    &    3. &   Det fremgår af GUI at data er gemt\\ \midrule	 	
 Undtagelser &    & [\textit{Undtagelse 1:}] Forsker trykker på Stop-knap\\
 & 1. 	& Systemet gemmer ufiltreret signal i tidsintervallet mellem tryk på Gem- og Stop-knap\\  
 & 2. & Det fremgår af GUI at data er gemt\\
 \\ \bottomrule
\caption{Fully dressed Use Case 4.}
\label{UC4}
\end{longtabu}

\begin{longtabu} to \linewidth{@{}l r X[j]@{}} %UC2%
	{\large \textbf{Use Case 5}} && \\
	\toprule
	Scenarie && Hovedscenarie\\
	Navn && Stop \\
	Mål && At stoppe måling\\
	Initiering && Forsker\\
	Aktører && Forsker (primær)\\
	Referencer && Use Case 1 -3  \\
	Samtidige forekomster  &&  pr. måling\\
	Forudsætninger && Use Case 1  - 3 er kørt succesfuldt\\
	Resultat && Måling stopper\\ \midrule
	Hovedscenarie &    1. &		Forsker trykker på Stop-knap\\[-1ex] 
	&    2. &   Måling stopper og systemet lukker\\ \midrule	 	
	Undtagelser &    & - 	\\ \bottomrule
	\caption{Fully dressed Use Case 5}
	\label{UC5}
\end{longtabu}

\section{Ikke-funktionelle krav}


\subsection{(F)URPS+}
MoSCoW er angivet i parentes med hhv. M, S, C eller W.

\textbf{Usability}


\textbf{Reliability}

				\begin{align}
					Availability = \frac{MTBF}{MTBF+MTTR} = \frac{20}{20+1} = 0,952 = 95,2 \%
				\end{align}



\textbf{Performance}

\textbf{Supportability}
















