\chapter{Kravspecifikation}

\begin{longtabu} to \linewidth{@{}l l l X[j]@{}}
    Version &    Dato &    Ansvarlig &    Beskrivelse\\[-1ex]
    \midrule
    1.0 &    21-09-2015 &    XXX &    Påbegyndt tilrettelse af UC1 og UC2
    
\label{version_Systemark}
\end{longtabu}


\section{Indledning}


\section{Funktionelle krav}
 

\subsection{Aktør-kontekstdiagram}

\begin{figure}[htb]
	\centering
	%\includegraphics[width=1\textwidth]{Figurer/Snip20150409_30}
	\caption{Aktør-kontekstdiagram}
	\label{fig:aktoerbeskrivelse}
\end{figure}



\subsection{Aktørbeskrivelse}

\begin{table}[H]
\begin{tabularx}{\textwidth}{l l X}
     Aktørnavn  & Type      & Beskrivelse \\ \midrule
     Bruger   & Primær    & Brugeren er den aktør, der ønsker at foretage målingerne, som omfatter EKG samt diagnosticering af artieflimmer. Brugeren er en person, der har kendskab til EKG-systemet. Fx sundhedsfaglig personale \\ 						  									  \addlinespace[2mm]
     
   
     \bottomrule                                                                                                                   
    \end{tabularx}
    \caption {Aktørbeskrivelse}
    \label{tab:aktoerbeskrivelse}
	
\end{table}

\subsection{Use case-diagram}

\begin{figure}[H]
	\centering
	%\includegraphics[width=1\textwidth]{Figurer/Snip20150518_11}
	\caption{Use case-diagram}
	\label{fig:Use Cases}
\end{figure}



\subsection{Use Cases}

\begin{longtabu} to \linewidth{@{}l r X[j]@{}} %UC1%
    {\large \textbf{Use Case 1}} && \\
    \toprule
    Scenarie && Hovedscenarie\\
    Navn && Start Måling\\
    Mål && At få foretaget en blodtryksmåling\\
    Initiering && Startes af Forsker\\
    Aktører && Forsker (primær), Patient(sekundær)\\
    Referencer && ???\\
    Samtidige forekomster  && Én patient pr. måling\\
    Forudsætninger && Alle systemer kører og Patient er klar til at få foretaget en måling\\ 
    Resultat && Patientens systole, diastole og puls bliver vist på GUI\\ \midrule
    Hovedscenarie &    1. &		Forsker indtaster patientoplysninger:\\[-1ex] 
   & &	a. Patient ID\\[-1ex]
   & &  b. Køn\\[-1ex]
   & &  c. Fødselsår\\[-1ex]				 	
                &    2. & Forsker trykker på Start-knap på GUI\\[-1ex]
                &    3. & Systolisk og diastolisk blodtryksmåling bliver vist kontinuerligt i GUI\\[-1ex]
                &    4. & Puls bliver vist i GUI\\ \midrule
                
    Undtagelser &    & \\ \bottomrule
\caption{Fully dressed Use Case 1.}
\label{UC1}
\end{longtabu}

\begin{longtabu} to \linewidth{@{}l r X[j]@{}} %UC2%
    {\large \textbf{Use Case 2}} && \\
    \toprule
    Scenarie && Hovedscenarie\\
    Navn && Analysér\\
    Mål && At vide om signalerne ligger inden for grænseværdierne\\
    Initiering && Startes af Forsker\\
    Aktører && Forsker (primær)\\
    Referencer && Use Case 1\\
    Samtidige forekomster  && Én patient pr. måling\\
    Forudsætninger && Use Case 1 er kørt succesfuldt\\
    Resultat && Patientens systoliske og diastoliske blodtryk og puls er blevet analyseret indenfor grænseværdierne\\ \midrule
    Hovedscenarie &    1. &		Forsker trykker på Analysér-knap\\[-1ex] 
                  &    2. &     Besked udskrives på GUI, hvis signalerne falder udenfor grænseværdierne\\ \midrule	 	
 Undtagelser &    & \\ \bottomrule
\caption{Fully dressed Use Case 2.}
\label{UC2}
\end{longtabu}

\begin{longtabu} to \linewidth{@{}l r X[j]@{}} %UC2%
    {\large \textbf{Use Case 3}} && \\
    \toprule
    Scenarie && Hovedscenarie\\
    Navn && Gam data\\
    Mål && At slutte og gemme data\\
    Initiering && Startes af Forsker\\
    Aktører && Forsker (primær), Database(sekundær)\\
    Referencer && Use Case 1\\
    Samtidige forekomster  && Én patient pr. måling\\
    Forudsætninger && Use Case 1 er kørt succesfuldt\\
    Resultat && Patientens systoliske og diastoliske blodtryk, puls og patientoplysninger er blevet gemt i en Database\\ \midrule
    Hovedscenarie &    3. &		Forsker trykker på Gem-knap\\[-1ex] 
                  &    4. &     Patientens systoliske og diastoliske blodtryk, puls og patientoplysninger er blevet gemt i en Database\\
                    &    5. &     5.	Det fremgår af GUI at data er gemt\\ \midrule	 	
 Undtagelser &    & \\ \bottomrule

\end{longtabu}


\section{Ikke-funktionelle krav}


\subsection{(F)URPS+}
MoSCoW er angivet i parentes med hhv. M, S, C eller W.

\textbf{Usability}


\textbf{Reliability}

				\begin{align}
					Availability = \frac{MTBF}{MTBF+MTTR} = \frac{20}{20+1} = 0,952 = 95,2 \%
				\end{align}



\textbf{Performance}

\textbf{Supportability}
















