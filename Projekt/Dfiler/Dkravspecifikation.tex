\chapter{Kravspecifikation}


\section{Versionshistorik}
\begin{longtabu} to \linewidth{@{}l l l X[j]@{}}
    Version 	&    Dato 		&    Ansvarlig 	&    Beskrivelse\\[-1ex]
    \midrule
    0.1 		&  	21-09-2015 	&   MHNK og MB 	&   Oprettelse og udfyldning af kravspecifikation \\
	0.2			&	24-09-2015	&	DHC og ABH	&	Omskrivning af UC1 - UC5 \\
	0.3			&	28-09-2015	&	ABH			&	Ikke-funktionelle krav \\
	0.4			&	08-10-2015	&	Alle		&	Tilrette efter review med Grp. 1 \\
	0.5			&	15-10-2015	&	MB			&	Indskrevet i LaTex \\
	0.6			&	20-10-2015	&	MHNK		&	Tilretning \\
    
\label{version_Systemark}
\end{longtabu}

\section{Godkendelsesformular}
\begin{longtabu} to \linewidth{@{}l X[j]@{}}
	Forfattere	&	Anne Hoelgaard, Ditte Heebøll, Martin Banasik, Albert Fredshavn, Mathias Munk og Mette Hammer Nielsen-Kudsk \\
	\midrule
	Godkendes af & Peter Johansen \\
	Antal sider & ??? \\
	Kunde	&	IHA \\
\end{longtabu}

Ved underskrivelse af dette dokument accepteres det af begge parter, som værende kravene til udviklingen af det ønskede system.
\\
\\
\\
\\
\noindent \begin{tabular}{lll} 
	& 	\makebox[2.5in]{\hrulefill} 	& 	\makebox[2.5in]{\hrulefill}\\
	&	Sted						&	Dato\\[7ex]
	& 	\makebox[2.5in]{\hrulefill} 	& 	\makebox[2.5in]{\hrulefill}\\
	& 	Kundens underskrift 		& 	Leverandørens underskrift\\[7ex]

\end{tabular}
\section{Indledning}
På baggrund af krav fra kunden samt hvad leverandøren finder muligt, er denne kravspecifikation blevet udarbejdet. Denne kravspecifikation har til formål at specificere kravene til produktet. Dette projekt tager udgangspunkt i en blodtryksmåler, hvortil der er en række aktører, som interagerer med et system, der er beskrevet yderligere nedenfor.

\section{Systembeskrivelse}
Blodtryksmåler apparatet ønskes udviklet således at systolisk og diastolisk blodtryk samt puls kan bestemmes ud fra en invasiv ateriel blodtryksmåling. Der udvikles instrumentering til den udleverede transducer som hardware og et software-program til visning af målt blodtryk. Disse to dele udgør til sammen vores system. 

\section{Funktionelle krav}
 

\subsection{Aktør-kontekstdiagram}


\begin{figure}[htb]
	\centering
	\includegraphics[width=0.8\textwidth]{Figurer/Aktor-kontekst-diagram}
	\caption{Aktør-kontekstdiagram}
	\label{fig:aktoerbeskrivelse}
\end{figure}
Af dette diagram ses vores aktører til at være: Forsker, Måleobjekt og Database. Herunder er der en detaljeret beskrivelse af hver aktør.


\subsection{Aktørbeskrivelse}

\begin{table}[H]
\begin{tabularx}{\textwidth}{l l X}
     Aktørnavn	&	Type		&	Beskrivelse \\ \midrule
     Forsker   	& 	Primær  	& 	Forskeren er aktøren, der starter og giver besked om at data ønskes gemt samt afslutter fysisk måling af blodtryk. \\ 			  \addlinespace[2mm]
     Database	&	Sekundær	&	Databasen er aktøren, hvori måledata bliver gemt. \\   \addlinespace[2mm]
     Måleobjekt	&	Sekundær	&	Måleobjekt er aktøren, hvorfra signalet og måledata indhentes. Dette kan være In Vitro maskinen, som findes i Cave Lab. Under løbende test i udviklingsprocessen benyttes Analog Discovery og PhysioNet. \\   \addlinespace[2mm]
     
   
     \bottomrule                                                                                                                   
    \end{tabularx}
    \caption {Aktørbeskrivelse}
    \label{tab:aktoerbeskrivelse}
	
\end{table}

\subsection{Use case-diagram}

\begin{figure}[H]
	\centering
	\includegraphics[width=0.8\textwidth]{Figurer/UseCasediagram}
	\caption{Use Case-diagram}
	\label{fig:Use Cases}
\end{figure}

Diagrammet ovenfor viser at systemet har fem Use Cases: Foretag nulpunktsjustering, Foretag kalibrering, Start måling, Gem data og Afslut måling. Herunder følger en nærmere beskrivelse af de enkelte Use Cases, gennem et fully-dressed Use Case skema. 
Systemet består af en computer, hvor vores software er placeret, en NI-DAQ, en Analog Discovery samt en transducer med tilhørende implementering. 
Brugergrænseflade er det som forsker initierer med, altså herfra hvor systemet aktiveres. Det forkortes til GUI. Grænseværdierne: SKAL SKRIVES UDFØRLIGT ET ANDET STED…..TILGÅR!


\newpage
\subsection{Use Cases}

\begin{longtabu} to \linewidth{@{}l r X[j]@{}} %UC1%
	{\large \textbf{Use Case 1}} && \\
	\toprule
	Scenarie 				&&	Hovedscenarie\\
	Navn 					&& 	Foretag nulpunktsjustering\\
	Mål 					&& 	At få foretaget en nulpunktsjustering\\
	Initiering 				&& 	Startes af Forsker\\
	Aktører 				&& 	Forsker (primær), Måleobjekt (sekundær)\\
	Referencer 				&& 	\\
	Samtidige forekomster  	&& 	Én nulpunktsjustering pr. kørsel \\
	Forudsætninger 			&&	Alle systemer er ledige og operationelle\\ 
	Resultat 				&& 	Nulpunktsjustering er blevet fortaget efter ønske\\ \midrule
	Hovedscenarie 			&    1. 	&	Pop-up vindue for nulpunktsjustering er åbent\\				 	
							&    2. 	& 	Forsker trykker på:\\ 
							& 			&	a. Ja-knap for at starte en nulpunktsjustering\\[-1ex]
							& 			&		 i. Nulpunktsjustering foretages og vinduet lukker efter endt justering\\[-1ex]
							& 			&  	b. Nej-knap for ikke at få foretaget en nulpunktsjustering\\[-1ex]
							& 			&   ii. Vinduet lukker\\[-1ex]	
	Undtagelser 			&			& 	-  \\ \bottomrule
	
	\caption{Fully dressed Use Case 1}
	\label{UC1}
\end{longtabu}

\begin{longtabu} to \linewidth{@{}l r X[j]@{}} %UC2%
	{\large \textbf{Use Case 2}} && \\
	\toprule
	Scenarie && Hovedscenarie\\
	Navn && Foretag kalibrering\\
	Mål && At få foretaget et valg om kalibrering, samt evt. udført kalibrering\\
	Initiering && Startes af Forsker\\
	Aktører && Forsker (primær), Måleobjekt (sekundær)\\
	Referencer && Use Case 1\\
	Samtidige forekomster  && Én kalibrering pr. kørsel \\
	Forudsætninger && Alle systemer er ledige og operationelle\\ 
	Resultat && Kalibrering er blevet foretaget efter ønske \\ \midrule
	Hovedscenarie &    1. &		Pop-up vindue for kalibrering er åbent\\				 	
	&    2. & Forsker trykker på:\\ 
	& &	a. Ja-knap for at starte en kalibrering\\[-1ex]
	& &		 i. Kalibrering foretages og vinduet lukker efter endt kalibrering\\[-1ex]
	& &  b. Nej-knap for ikke at få foretaget en kalibrering\\[-1ex]
	& &   ii. Vinduet lukker\\[-1ex]	
	Undtagelser && -  \\ \bottomrule
	
	\caption{Fully dressed Use Case 2}
	\label{UC2}
\end{longtabu}

\begin{longtabu} to \linewidth{@{}l r X[j]@{}} %UC3%
    {\large \textbf{Use Case 3}} && \\
    \toprule
    Scenarie && Hovedscenarie\\
    Navn && Start Måling\\
    Mål && At få foretaget en blodtryksmåling\\
    Initiering && Startes af Forsker\\
    Aktører && Forsker (primær), Måleobjekt (sekundær)\\
    Referencer && Use Case 1 og Use Case 2\\
    Samtidige forekomster  && Ét signal pr. måling\\
    Forudsætninger && Use Case 1 og Use Case 2 er kørt succesfuldt, samt alle systemer kører og er klar til at foretage en måling\\ 
    Resultat && Systolisk-, diastolisk blodtryk samt puls bliver vist på GUI\\ \midrule
    Hovedscenarie &    1. &		Forsker indtaster Forsøgsnavn\\[-1ex]
    &     2. & 	Filteret signal er valgt per default af systemet\\	 	
                &    & [\textit{Undtagelse 1:}] Forsker vælger ufiltreret signal på radiobutton	\\
                &    3. & Forsker trykker på Start-knap på GUI\\[-1ex]
                &    4. & Signal for blodtryk vises på GUI\\[-1ex] 
                &    5. & Systolisk og diastolisk blodtryk samt puls bliver vist i bokse på GUI\\ \midrule
                
    Undtagelser && [\textit{Undtagelse 1}]	Forsker vælger ufiltreret signal\\ 
    &	a. & Use Case fortsættes fra punkt 3\\ 
    \bottomrule
\caption{Fully dressed Use Case 3}
\label{UC3}
\end{longtabu}


\begin{longtabu} to \linewidth{@{}l r X[j]@{}} %UC4%
    {\large \textbf{Use Case 4}} && \\
    \toprule
    Scenarie && Hovedscenarie\\
    Navn && Gem data\\
    Mål && At gemme data i databasen\\
    Initiering && Startes af Forsker\\
    Aktører && Forsker (primær), Database(sekundær)\\
    Referencer && Use Case 1 - 3 \\
    Samtidige forekomster  &&  Ét signal pr. måling\\
    Forudsætninger && Use Case 1 og 2 er kørt succesfuldt, Use Case 3 kører\\
    Resultat && Systolisk- og diastolisk blodtryk samt puls er blevet gemt i en Database under Forsøgsnavn\\ \midrule
    Hovedscenarie &    1. &		Forsker trykker på Gem-knap\\[-1ex] 
    &	2. & Systemet gemmer det fremadrettede ufiltreret signal i Database\\
                  &    3. &  Forsker trykker på Gem-knap for at stoppe med at gemme\\
    &	&			[\textit{Undtagelse 1:}] Forsker trykker på Afslut-knap\\
                    &    4. &   Det fremgår af GUI at data er gemt\\ \midrule	 	
 Undtagelser &    & [\textit{Undtagelse 1:}] Forsker trykker på Afslut-knap\\
 & a. 	&  Systemet stopper med at gemme, hvorefter systemet lukker\\  

 \\ \bottomrule
\caption{Fully dressed Use Case 4}
\label{UC4}
\end{longtabu}
\newpage
\begin{longtabu} to \linewidth{@{}l r X[j]@{}} %UC5%
	{\large \textbf{Use Case 5}} && \\
	\toprule
	Scenarie && Hovedscenarie\\
	Navn && Afslut måling \\
	Mål && At stoppe måling af blodtryk\\
	Initiering && Startes af Forsker\\
	Aktører && Forsker (primær)\\
	Referencer && Use Case 1 -3  \\
	Samtidige forekomster  &&  Ét signal pr. måling\\
	Forudsætninger && Use Case 1 og 2 er kørt succesfuldt, Use Case 3 kører\\
	Resultat && Måling af blevet stoppet\\ \midrule
	Hovedscenarie &    1. &		Forsker trykker på Afslut-knap\\[-1ex] 
	&    2. &   Måling stopper, hvorefter systemet lukker. \\ \midrule	 	
	Undtagelser &    & - 	\\ \bottomrule
	\caption{Fully dressed Use Case 5}
	\label{UC5}
\end{longtabu}

\section{Ikke-funktionelle krav}


\subsection{(F)URPS+}

\textbf{Functionality}
\begin{enumerate}
\item Blodtryksmåleren skal indeholde en Start-knap til at igangsætte målingerne. 
\item Blodtryksmåleren skal indeholde en Afslut-knap hvor fra måling kan stoppes.
\item Blodtryksmåleren skal indeholde en Gem-knap til at gemme målingerne i Databasen.
\item Blodtryksmåleren skal indeholde en tekstboks til forsøgsnavn, hvori forsker indtaster det pågældende forsøgsnavn.
\item Blodtryksmåleren skal indeholde radiobutton til filtreret signal, denne skal være default valget.
\item Blodtryksmåleren skal indeholde radiobutton til ufiltreret signal.
\item Blodtryksmåleren skal indeholde tekstbokse til puls, systolisk og diastolisk blodtryk som vises med op til tre cifre.
\item GUI’en skal se ud som vist på figur xxx:    INDSÆT SKITSE AF GUI!
\end{enumerate}

\textbf{Usability}
\begin{enumerate}
\item Forskeren skal kunne starte en default-måling maksimalt 30 sekunder efter systemet er startet.
\end{enumerate}

\textbf{Reliability}
\begin{enumerate}
\item Det skal maksimalt tage 5 timer at gendanne systemet (MTTR - Mean Time To Restore)
\item Systemet skal have en oppetid uden nedbrud på minimum 1 måned (720 timer) (MTBF - Mean Time Between Failure).   
\item Systemet skal have en oppetid/køretid på: 
\end{enumerate}

				\begin{align}
					Availability = \frac{MTBF}{MTBF+MTTR}*100 = \frac{720}{720+5}*100 = 99,31 \%
				\end{align}



\textbf{Performance}
\begin{enumerate}
\item Blodtryksmåleren skal, indenfor 3 sekunder, kunne vise systolisk og diastolisk blodtryk via graf. Dette accepteres med en tolerance på +/- 15 \%.
\item Blodtryksmåleren skal, inden for måleperioden, kunne alarmere hvis patienten har forhøjet eller for lavt blodtryk. Defineres efter grænseværdier beskrevet i… xxxx…
\item Blodtryksmåleren skal, indenfor 30 sekunder fra der er trykket på Gem-knap, kunne gemme målingerne i Databasen.  Dette accepteres med en tolerance på +/- 15 \%.
\item Grafen vises i ét vindue, hvor y-aksen måles i mmHg (millimeter kviksølv) og x-aksen i tid pr. sekund. 
\item Hvert 7. sekund skal værdier for systolisk og diastolisk blodtryk samt puls opdateres. Dette accepteres med en tolerance på +/- 15 \%.
\item Graf for blodtryk skal køre kontinuerligt i GUI efter følgende princip: \textbf{INDSÆT FIGUR og beskrivelse AF GRAF!}
\item Når der trykkes på Gem-knap gemmes det ufiltrerede signal under det indtastede forsøgsnavn og et autogenereret nr. \textit{"forsøgsnavn\_nr"}
\item Systemet skal kunne måle blodtryksværdier fra 0 til 250 mmHg.
\end{enumerate}


\textbf{Supportability}
\begin{enumerate}
\item Forskeren skal kunne udskifte hardwaren på 10 minutter. 
\item Softwaren skal opbygges med lav kobling. 
\end{enumerate}
















