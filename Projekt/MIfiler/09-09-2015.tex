\chapter{Mødeindkaldelse}


\section{Dagsorden: 09-09-2015}
\hrule



\textbf{Mødetidspunkt/-sted:} 

Dato: \tabto{7em} 09-09-2015

Tidspunkt: \tabto{7em} kl. 09.00

Sted: \tabto{7em} 204E

Mødetype: \tabto{7em} Vejledermøde \newline


\hrule
\textbf{Opgaver:} \newline
Formålet med mødet er, at få gennemgået projektet med vejleder Peter Johansen.
\begin{enumerate}
\item Gennemgang af samarbejdskontrakt

\item Logbog/referat - Hvordan skal det udføres?

\item Tidsplan for projektet - Deadlines med vejleder?
\begin{enumerate}
\item Hvor skal vi starte?
\end{enumerate}
\item Hvilke problemer kan Peter hjælpe med og hvilke typer problemer skal vi gå til andre vejleder med?
\item Aftale fast mødedag med vejleder
\end{enumerate}

\newpage
\section{Dagsorden: 15-09-2015}
\hrule



\textbf{Mødetidspunkt/-sted:} 

Dato: \tabto{7em} 15-09-2015

Tidspunkt: \tabto{7em} kl. 12.15

Sted: \tabto{7em} 408E

Mødetype: \tabto{7em} Gruppemøde \newline


\hrule
\textbf{Opgaver:} \newline
Formålet med mødet er, at få klarlagt ønskede krav og funktioner til produktets hardware og software.
\begin{enumerate}
\item Brainstorm på funktioner og krav til produktet, samt udvælgelse af hvilke vi vil arbejde videre med.

\item Kravspecifikation - Herunder:

\begin{enumerate}
\item Aktør-kontekst diagram
\end{enumerate}
\begin{enumerate}
\item Funktionelle krav - Use Case
\end{enumerate}
\begin{enumerate}
\item Ikke-funktionelle krav
\end{enumerate}
\item Accepttest - Funktionelle og ikke-funktionelle krav
\end{enumerate}

Inden mødet vil det være fint, hvis vi hver især har overvejet hvilke funktioner og krav der er relevante i forhold til forsker-vinklen, såfremt vi holder fast i denne vinkel.