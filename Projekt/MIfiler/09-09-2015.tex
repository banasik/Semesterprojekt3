\chapter{Mødeindkaldelse}


\section{Dagsorden: 09-09-2015}
\hrule



\textbf{Mødetidspunkt/-sted:} 

Dato: \tabto{7em} 09-09-2015

Tidspunkt: \tabto{7em} kl. 09.00

Sted: \tabto{7em} 204E

Mødetype: \tabto{7em} Vejledermøde \newline


\hrule
\textbf{Opgaver:} \newline
Formålet med mødet er, at få gennemgået projektet med vejleder Peter Johansen.
\begin{enumerate}
\item Gennemgang af samarbejdskontrakt

\item Logbog/referat - Hvordan skal det udføres?

\item Tidsplan for projektet - Deadlines med vejleder?
\begin{enumerate}
\item Hvor skal vi starte?
\end{enumerate}
\item Hvilke problemer kan Peter hjælpe med og hvilke typer problemer skal vi gå til andre vejleder med?
\item Aftale fast mødedag med vejleder
\end{enumerate}

\newpage
\section{Dagsorden: 15-09-2015}
\hrule



\textbf{Mødetidspunkt/-sted:} 

Dato: \tabto{7em} 15-09-2015

Tidspunkt: \tabto{7em} kl. 12.15

Sted: \tabto{7em} 408E

Mødetype: \tabto{7em} Gruppemøde \newline


\hrule
\textbf{Opgaver:} \newline
Formålet med mødet er, at få klarlagt ønskede krav og funktioner til produktets hardware og software.

Brainstorm på funktioner og krav til produktet, samt udvælgelse af hvilke vi vil arbejde videre med.


Kravspecifikation - Herunder:
\begin{enumerate}
\item Aktør-kontekst diagram
\item Funktionelle krav - Use Case
\item Ikke-funktionelle krav
\item Accepttest - Funktionelle og ikke-funktionelle krav
\end{enumerate}

Inden mødet vil det være fint, hvis vi hver især har overvejet hvilke funktioner og krav der er relevante i forhold til forsker-vinklen, såfremt vi holder fast i denne vinkel.

\newpage
\section{Dagsorden: 21-09-2015}
\hrule



\textbf{Mødetidspunkt/-sted:} 

Dato: \tabto{7em} 21-09-2015

Tidspunkt: \tabto{7em} kl. 10:15 (mellemtimer)

Sted: \tabto{7em} 408E

Mødetype: \tabto{7em} Gruppemøde \newline


\hrule
\textbf{Opgaver:} \newline
Formålet med mødet er, at få påbegyndt udarbejdelse af kravspecifikation og accepttest.
\begin{enumerate}
\item Gennemgang af svaret fra Samuel - Har vi ikke modtaget svar, forsøges Samuel opsøgt på skolen ellers arbejder vi videre ud fra vores bedste overbevisning.

\item Kravspecifikation - Herunder:

\begin{enumerate}
\item Aktør-kontekst diagram
\item Funktionelle krav - Use Case
\item Ikke-funktionelle krav
<<<<<<< HEAD
\end{enumerate}
\item Accepttest - Funktionelle og ikke-funktionelle krav
\end{enumerate}


\newpage
\section{Dagsorden: 24-09-2015}
\hrule



\textbf{Mødetidspunkt/-sted:} 

Dato: \tabto{7em} 24-09-2015

Tidspunkt: \tabto{7em} kl. 08:30 

Sted: \tabto{7em} 204E

Mødetype: \tabto{7em} Vejledermøde \newline


\hrule
\textbf{Opgaver:} \newline
Formålet med mødet er, at få afklaret en række spørgsmål med vejleder om udarbejdelsen af kravspecifikation.
Udkast til kravspecifikation og accepttest er sendt til Peter inden mødestart.
\begin{enumerate}
\item Spørgsmål til vejleder - Vil blive uddybet yderligere til mødet:
\begin{enumerate}
\item Skelne tydeligt mellem prototype og final-produkt?
\item Zoom ind på graf
\item Pause-knap
\item Analyser-Use Case: Skal den implementeres som pop-up, label eller blot gemmes i databasen?
\item Gem data løbende eller efter endt måling?
\item Muligt at udskrive råt og filtreret signal samtidig? Relevant?
\item Konstant eller svingende puls?
\end{enumerate}
\end{enumerate}

\newpage
\section{Dagsorden: 24-09-2015}
\hrule



\textbf{Mødetidspunkt/-sted:} 

Dato: \tabto{7em} 24-09-2015

Tidspunkt: \tabto{7em} kl. 14:15

Sted: \tabto{7em} 408E

Mødetype: \tabto{7em} Gruppemøde \newline


\hrule
\textbf{Opgaver:} \newline
Formålet med mødet er, at arbejde videre med kravspecifikation og accepttest.
\begin{enumerate}
\item Ændre og tilrette kravspecifikation mht. konklusionerne fra vejledermødet.

\item Ændre og tilrette accepttest

\item LaTeX - Status på oprettelse af generelle dokumenter til rapporten?

\item Gennemgang af tidsplan og tilføjelse af interne deadlines. Følgende punkter skal have et begyndelsestidspunkt samt deadline:
\begin{enumerate}
\item Officielle deadlines
\item Definering af forsknings-case
\item Projektformulering, kravspek. og accepttest
\item Software - Koden
\item Hardware
\item Hardware og software design
\item Korrekturlæsning
\item Print
\item Flere ???

\end{enumerate}
\end{enumerate}

\newpage
\section{Dagsorden: 29-09-2015}
\hrule



\textbf{Mødetidspunkt/-sted:} 

Dato: \tabto{7em} 29-09-2015

Tidspunkt: \tabto{7em} kl. 12.15

Sted: \tabto{7em} 408E

Mødetype: \tabto{7em} Gruppemøde \newline


\hrule
\textbf{Opgaver:} \newline
Formålet med mødet er, at få færdiggjort kravspecifikation og accepttest fuldstændig, samt påbegynde software-del.
\begin{enumerate}
\item Kommentarer/spørgsmål til Kravspecifikation og accepttest

\item Kl. 13:00 - Vejledermøde med Peter, lokale 204E

\item Påbegynde programmering (se nedestående forslag)

\item Evt. start på design af GUI
\end{enumerate}

Et forslag går på at vi skal nå frem til en fælles forståelse af hvordan vi griber programmeringen an. Dette kan gøres ved at vi deler os op i mindre grupper (2-3 personer) som tildeles et kode-element, hver gruppe snakker sig så frem til hvad netop denne lille bid af koden skal kunne  og mulige måder til hvordan det kan implementeres. Det vil aktivere flere i processen og Så har vi en række forslag inden selve programmeringen påbegyndes.

\newpage
\section{Dagsorden: 06-10-2015}
\hrule



\textbf{Mødetidspunkt/-sted:} 

Dato: \tabto{7em} 06-10-2015

Tidspunkt: \tabto{7em} kl. 12.15

Sted: \tabto{7em} 408E

Mødetype: \tabto{7em} Gruppemøde \newline


\hrule
\textbf{Opgaver:} \newline
Formålet med mødet er, at gennemgå Grp. 1's arbejde samt hvordan vi kommer videre fra hvor vi står nu.
\begin{enumerate}
\item Gennemgang af Grp. 1's kravspek. og accepttest - Sørg for at have læst denne igennem hjemmefra.

\item Skal vi opdeles i en Software og Hardware gruppe? Ansvarsområder - mindre opgaver.

\item Udkast til GUI

\item Evt.? Spørgsmål til vejleder.
\end{enumerate}

\newpage
\section{Dagsorden: 20-10-2015}
\hrule



\textbf{Mødetidspunkt/-sted:} 

Dato: \tabto{7em} 20-10-2015

Tidspunkt: \tabto{7em} kl. 12.15

Sted: \tabto{7em} 408E

Mødetype: \tabto{7em} Gruppemøde \newline


\hrule
\textbf{Opgaver:} \newline
Formålet med mødet er, at de to undergrupper arbejder videre med software og hardware.
\begin{enumerate}
\item Opsamling efter ferien:
\begin{enumerate}
\item Har vi fået svar fra Peter?
\item Nogen der har arbejdet med noget i ferien?
\end{enumerate}

\item Tjekke kravspek. og accepttest igennem

\item Gennemgang af brugen af DAQ ved Brian

\item Videre arbejde med software og hardware i undergrupper

\item Til slut: Status på software og hardware udvikling
\end{enumerate}


\newpage
\section{Dagsorden: 27-10-2015}
\hrule



\textbf{Mødetidspunkt/-sted:} 

Dato: \tabto{7em} 27-10-2015

Tidspunkt: \tabto{7em} kl. 12.15

Sted: \tabto{7em} 408E

Mødetype: \tabto{7em} Gruppemøde \newline


\hrule
\textbf{Opgaver:} \newline
Formålet med mødet er, at de to undergrupper arbejder videre med software og hardware. Samt opstart på ISE-diagrammer.
\begin{enumerate}
\item Status fra Hardware-gruppe

\item Status fra Software-gruppe

\item Begynde på ISE-diagrammer - Tag udgangspunkt i to do-listen der blev udarbejdet på sidste gruppemøde
\begin{enumerate}
\item Overordnet sekvensdiagram
\item Domænemodel
\item Applikationsmodel
\item BDD og IBD (Husk komponentliste)
\end{enumerate}

\item Vejledermøde med Peter på onsdag rykkes til 08:30?
\end{enumerate}