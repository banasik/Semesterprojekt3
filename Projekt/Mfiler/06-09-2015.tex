\chapter{Mødereferat}

\section{Dato: 09-09-2015}
\hrule

\textbf{Fremmødte:} Anne, Ditte, Mette, Albert, Martin og Mathias

\textbf{Fraværende:} Ingen

\textbf{Referent:} Mathias Munk

\textbf{Dagens dagsorden:}

	\textbf{Opgaver:} \newline
Formålet med mødet er, at få gennemgået projektet med vejleder Peter Johansen.
\begin{enumerate}
\item Gennemgang af samarbejdskontrakt

\item Logbog/referat - Hvordan skal det udføres?

\item Tidsplan for projektet - Deadlines med vejleder?
\begin{enumerate}
\item Hvor skal vi starte?
\end{enumerate}
\item Hvilke problemer kan Peter hjælpe med og hvilke typer problemer skal vi gå til andre vejleder med?
\item Aftale fast mødedag med vejleder
\end{enumerate}


\textbf{Referat:}

\begin{enumerate}
\item Den er fin som den er. Skal printes ud, underskrives og scannes ind igen.
\item Logbog skal være til status. Referat skrives til hvert vejleder møde og andre møder.
\begin{enumerate}
\item Logbog
\begin{enumerate}
\item Når der er arbejdsmøde og ikke møder med vejleder, hvor der skal dokumenteres hvad der er blevet gennemført og hvad der mangler at blive gjort 
\end{enumerate}
\end{enumerate}
\item Vi skal hurtigst muligt lave et udkast til accepttest og kravspecifikation samt tidsplan. Derefter skal der foregå en frem-og-tilbage kontakt med vejleder.
\begin{enumerate}
\item Kravspek. Accepttest.
\begin{enumerate}
\item Accepttest bliver meget simulering
\item Projektet kan gå i to retninger
\begin{enumerate}
\item Klinisk forskning
\item Forskningsprojekt
\end{enumerate}
\end{enumerate}
\end{enumerate}
\item KVI - blodtryksmåling og transducer
\begin{enumerate}
\item Hvad kan han hjælpe med
\begin{enumerate}
\item Blodtryksmåling og transducer
\item Han kan godt forstå logiske ISE diagrammer
\end{enumerate}
\item Hvad kan han ikke hjælpe med
\begin{enumerate}
\item Programmering - gå til Lars Mortensen f.eks.
\end{enumerate}
\end{enumerate}
\item Hver onsdag kl.09.
\begin{enumerate}
\item Onsdag d.16 og onsdag d.30 kan han ikke.
\end{enumerate}
\item Angående ISE diagrammer skal vi lave relevante diagrammer. Ingen grund til at lave state-machine bare for at lave et diagram.
\end{enumerate}



\section{Dato: 15-09-2015}
\hrule

\textbf{Fremmødte: Anne, Ditte, Mette, Martin, Albert og Mathias} 

\textbf{Fraværende:} Ingen

\textbf{Referent:} Mathias Munk

\textbf{Dagens dagsorden:}
\newline
Formålet med mødet er, at få klarlagt ønskede krav og funktioner til produktets hardware og software.
\begin{enumerate}
\item Brainstorm på funktioner og krav til produktet, samt udvælgelse af hvilke vi vil arbejde videre med.

\item Kravspecifikation - Herunder:

\begin{enumerate}
\item Aktør-kontekst diagram
\end{enumerate}
\begin{enumerate}
\item Funktionelle krav - Use Case
\end{enumerate}
\begin{enumerate}
\item Ikke-funktionelle krav
\end{enumerate}
\item Accepttest - Funktionelle og ikke-funktionelle krav
\end{enumerate}

Inden mødet vil det være fint, hvis vi hver især har overvejet hvilke funktioner og krav der er relevante i forhold til forsker-vinklen, såfremt vi holder fast i denne vinkel.


\textbf{Referat:}
\begin{enumerate}
\item Der er valgt at produktet skal have tilføjet, så der kan måle systolisk/diastolisk blodtryk, samt skal den måle puls. Der er ikke valgt om der skal tilføjes en biplyd. 
\begin{enumerate}
\item De målte data skal lagres i en database
\item Alarmering når blodtrykket stiger over en grænseværdi.
\begin{enumerate}
\item Hertil er der tænkt at der laves en måling af blodtryk hos en person med diabetes eller en ældre person. Det skulle gerne give et generelt højere blodtryksresultat. Disse resultater skal derefter sammenlignes med et blodtryk fra en ung og rask person.
\begin{enumerate}
\item Albert har skrevet en mail til Samuel om hvorvidt det er muligt at finde disse personer på PhysioBank.
\end{enumerate}
\end{enumerate}
\end{enumerate}
\item Der kan ikke laves noget specifikt nu
\begin{enumerate}
\item Alt bliver taget på mandag d. 21/9 hvor der afholdes nyt møde, som ligger før/efter KSS afhængig af hvornår KSS er.
\end{enumerate}
\item Der kan ikke laves noget specifikt nu
\begin{enumerate}
\item Bliver taget op til næste møde på mandag d.21/09 før/efter KSS
\end{enumerate}
\item Albert vil lægge svaret fra Samuel op i vores Facebook gruppe
\begin{enumerate}
\item Vi andre i gruppen skal følge op på Samuels svar
\end{enumerate}
\end{enumerate}



\section{Dato: 21-09-2015}
\hrule

\textbf{Fremmødte:} Martin, Mette, Ditte, Anne, Mathias

\textbf{Fraværende:} Albert

\textbf{Referent:} Mathias Munk

\textbf{Dagens dagsorden:}
\newline
Formålet med mødet er, at få påbegyndt udarbejdelse af kravspecifikation og accepttest.
\begin{enumerate}
\item Gennemgang af svaret fra Samuel - Har vi ikke modtaget svar, forsøges Samuel opsøgt på skolen ellers arbejder vi videre ud fra vores bedste overbevisning.

\item Kravspecifikation - Herunder:

\begin{enumerate}
\item Aktør-kontekst diagram
\item Funktionelle krav - Use Case
\item Ikke-funktionelle krav
\end{enumerate}
\item Accepttest - Funktionelle og ikke-funktionelle krav
\end{enumerate}

\textbf{Referat:}

\begin{enumerate}
\item Gennemgangen af Samuels svar er ikke fyldestgørende, da svaret fra ham er blevet for sent lagt op og det har resulteret i at den korrespondance der ville have været, ikke er der.
\item Kravspecifikationen bliver delt op
\begin{enumerate}
\item Aktør-kontekst diagram står Mathias for.
\item Funktionelle krav - Use Case står Anne og Ditte for.
\item Ikke-funktionelle krav står Martin og Mette for.
\end{enumerate}
\item Accepttesten for ikke-funktionelle krav vil Mette stå for. Der vil blive lavet et grovt udkast, som vil blive sendt til Ditte, der derefter sender det videre til Peter.
\item \textbf{Andet:}
\begin{enumerate}
\item Det er vedtaget at følgende ord formuleres udelukkende sådan gennem hele vores rapport, uanset hvilken sammenhæng de er skrevet i:
\begin{itemize}
\item Use Case
\item Database
\item Black Box - Altid to store B'er
\item Start-knap
\item Analysér-knap
\item Gem-knap
\end{itemize}
\end{enumerate}
\end{enumerate}

\section{Dato: 07-09-2015}
\hrule

\textbf{Fremmødte:} 

\textbf{Fraværende:}

\textbf{Referent:} Mathias Munk

\textbf{Dagens dagsorden:}
\begin{itemize}
	\item Emne
	\item Emne
	\item Emne
\end{itemize}

\textbf{Referat:}

TEKST TEKST TEKST
