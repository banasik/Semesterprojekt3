\chapter{Mødereferat}

\section{Dato: 09-09-2015}
\hrule

\textbf{Fremmødte:} Anne, Ditte, Mette, Albert, Martin og Mathias

\textbf{Fraværende:} Ingen

\textbf{Referent:} Mathias Munk

\textbf{Dagens dagsorden:}

	\textbf{Opgaver:} \newline
Formålet med mødet er, at få gennemgået projektet med vejleder Peter Johansen.
\begin{enumerate}
\item Gennemgang af samarbejdskontrakt

\item Logbog/referat - Hvordan skal det udføres?

\item Tidsplan for projektet - Deadlines med vejleder?
\begin{enumerate}
\item Hvor skal vi starte?
\end{enumerate}
\item Hvilke problemer kan Peter hjælpe med og hvilke typer problemer skal vi gå til andre vejleder med?
\item Aftale fast mødedag med vejleder
\end{enumerate}


\textbf{Referat:}

\begin{enumerate}
\item Den er fin som den er. Skal printes ud, underskrives og scannes ind igen.
\item Logbog skal være til status. Referat skrives til hvert vejleder møde og andre møder.
\begin{enumerate}
\item Logbog
\begin{enumerate}
\item Når der er arbejdsmøde og ikke møder med vejleder, hvor der skal dokumenteres hvad der er blevet gennemført og hvad der mangler at blive gjort 
\end{enumerate}
\end{enumerate}
\item Vi skal hurtigst muligt lave et udkast til accepttest og kravspecifikation samt tidsplan. Derefter skal der foregå en frem-og-tilbage kontakt med vejleder.
\begin{enumerate}
\item Kravspek. Accepttest.
\begin{enumerate}
\item Accepttest bliver meget simulering
\item Projektet kan gå i to retninger
\begin{enumerate}
\item Klinisk forskning
\item Forskningsprojekt
\end{enumerate}
\end{enumerate}
\end{enumerate}
\item KVI - blodtryksmåling og transducer
\begin{enumerate}
\item Hvad kan han hjælpe med
\begin{enumerate}
\item Blodtryksmåling og transducer
\item Han kan godt forstå logiske ISE diagrammer
\end{enumerate}
\item Hvad kan han ikke hjælpe med
\begin{enumerate}
\item Programmering - gå til Lars Mortensen f.eks.
\end{enumerate}
\end{enumerate}
\item Hver onsdag kl.09.
\begin{enumerate}
\item Onsdag d.16 og onsdag d.30 kan han ikke.
\end{enumerate}
\item Angående ISE diagrammer skal vi lave relevante diagrammer. Ingen grund til at lave state-machine bare for at lave et diagram.
\end{enumerate}



\section{Dato: 15-09-2015}
\hrule

\textbf{Fremmødte: Anne, Ditte, Mette, Martin, Albert og Mathias} 

\textbf{Fraværende:} Ingen

\textbf{Referent:} Mathias Munk

\textbf{Dagens dagsorden:}
\newline
Formålet med mødet er, at få klarlagt ønskede krav og funktioner til produktets hardware og software.
\begin{enumerate}
\item Brainstorm på funktioner og krav til produktet, samt udvælgelse af hvilke vi vil arbejde videre med.

\item Kravspecifikation - Herunder:

\begin{enumerate}
\item Aktør-kontekst diagram
\end{enumerate}
\begin{enumerate}
\item Funktionelle krav - Use Case
\end{enumerate}
\begin{enumerate}
\item Ikke-funktionelle krav
\end{enumerate}
\item Accepttest - Funktionelle og ikke-funktionelle krav
\end{enumerate}

Inden mødet vil det være fint, hvis vi hver især har overvejet hvilke funktioner og krav der er relevante i forhold til forsker-vinklen, såfremt vi holder fast i denne vinkel.


\textbf{Referat:}
\begin{enumerate}
\item Der er valgt at produktet skal have tilføjet, så der kan måle systolisk/diastolisk blodtryk, samt skal den måle puls. Der er ikke valgt om der skal tilføjes en biplyd. 
\begin{enumerate}
\item De målte data skal lagres i en database
\item Alarmering når blodtrykket stiger over en grænseværdi.
\begin{enumerate}
\item Hertil er der tænkt at der laves en måling af blodtryk hos en person med diabetes eller en ældre person. Det skulle gerne give et generelt højere blodtryksresultat. Disse resultater skal derefter sammenlignes med et blodtryk fra en ung og rask person.
\begin{enumerate}
\item Albert har skrevet en mail til Samuel om hvorvidt det er muligt at finde disse personer på PhysioBank.
\end{enumerate}
\end{enumerate}
\end{enumerate}
\item Der kan ikke laves noget specifikt nu
\begin{enumerate}
\item Alt bliver taget på mandag d. 21/9 hvor der afholdes nyt møde, som ligger før/efter KSS afhængig af hvornår KSS er.
\end{enumerate}
\item Der kan ikke laves noget specifikt nu
\begin{enumerate}
\item Bliver taget op til næste møde på mandag d.21/09 før/efter KSS
\end{enumerate}
\item Albert vil lægge svaret fra Samuel op i vores Facebook gruppe
\begin{enumerate}
\item Vi andre i gruppen skal følge op på Samuels svar
\end{enumerate}
\end{enumerate}



\section{Dato: 21-09-2015}
\hrule

\textbf{Fremmødte:} Martin, Mette, Ditte, Anne, Mathias

\textbf{Fraværende:} Albert

\textbf{Referent:} Mathias Munk

\textbf{Dagens dagsorden:}
\newline
Formålet med mødet er, at få påbegyndt udarbejdelse af kravspecifikation og accepttest.
\begin{enumerate}
\item Gennemgang af svaret fra Samuel - Har vi ikke modtaget svar, forsøges Samuel opsøgt på skolen ellers arbejder vi videre ud fra vores bedste overbevisning.

\item Kravspecifikation - Herunder:

\begin{enumerate}
\item Aktør-kontekst diagram
\item Funktionelle krav - Use Case
\item Ikke-funktionelle krav
\end{enumerate}
\item Accepttest - Funktionelle og ikke-funktionelle krav
\end{enumerate}

\textbf{Referat:}

\begin{enumerate}
\item Gennemgangen af Samuels svar er ikke fyldestgørende, da svaret fra ham er blevet for sent lagt op og det har resulteret i at den korrespondance der ville have været, ikke er der.
\item Kravspecifikationen bliver delt op
\begin{enumerate}
\item Aktør-kontekst diagram står Mathias for.
\item Funktionelle krav - Use Case står Anne og Ditte for.
\item Ikke-funktionelle krav står Martin og Mette for.
\end{enumerate}
\item Accepttesten for ikke-funktionelle krav vil Mette stå for. Der vil blive lavet et grovt udkast, som vil blive sendt til Ditte, der derefter sender det videre til Peter.
\item \textbf{Andet:}
\begin{enumerate}
\item Det er vedtaget at følgende ord formuleres udelukkende sådan gennem hele vores rapport, uanset hvilken sammenhæng de er skrevet i:
\begin{itemize}
\item Use Case
\item Database
\item Black Box - Altid to store B'er
\item Start-knap
\item Analysér-knap
\item Gem-knap
\end{itemize}
\end{enumerate}
\end{enumerate}

\section{Dato: 24-09-2015}
\hrule

\textbf{Fremmødte:} Martin, Mette, Ditte, Anne og Albert

\textbf{Fraværende:} Mathias

\textbf{Referent:} Ditte

\textbf{Dagens dagsorden:}
\newline
Formålet med mødet er, at få afklaret en række spørgsmål med vejleder om udarbejdelsen af kravspecifikation.
Udkast til kravspecifikation og accepttest er sendt til Peter inden mødestart.
\begin{enumerate}
\item Spørgsmål til vejleder - Vil blive uddybet yderligere til mødet:
\begin{enumerate}
\item Skelne tydeligt mellem prototype og final-produkt?
\item Zoom ind på graf
\item Pause-knap
\item Analyser-Use Case: Skal den implementeres som pop-up, label eller blot gemmes i databasen?
\item Gem data løbende eller efter endt måling?
\item Muligt at udskrive råt og filtreret signal samtidig? Relevant?
\item Konstant eller svingende puls?
\end{enumerate}
\end{enumerate}


\textbf{Referat:}
\begin{enumerate}
\item Vores produkt kommer aldrig til at være i forbindelse med en patient, da den skal bruges oppe i CAVE Lab. Den bliver derved bare en prototype.
\item Funktionen "zoom ind på graf" skal kun tilføjes hvis vi har tid.
\item Pause-knappen er droppet, Stop-knappen er kommet i stedet.
\item Analysér-Use Case er droppet, men dertil er der kommet flere Use Cases.
\item Når der trykkes på Gem-knappen, gemmes det kommende stykke data. Mængden af Data varierer efter ønske. Det skal være muligt at trykke Stop og derved gemme det der er målt.
\item Det rå data skal altid gemmes. Det der skal vises, skal kunne skiftes i mellem rå og filtreret. Som standard er det det filtrerede der skal vises.
\item Konstant puls. Puls, diastolisk og systolisk 	blodtryk skal hele tiden vises. Der bliver sat et interval på hvor ofte de tre værdier skal regnes ud. Peter vil gerne have 3 sekunder på intervallet, hvis tallet bliver højere er det mere usikkert. 
\item \textbf{Andet:}
\begin{enumerate}
\item Y-akse skal vise mmHg. X-aksen viser tid.
\item Inden start skal den spørge i forhold til nulpunktsjustering og/eller kalibrering.
\item Login er IKKE nødvendigt.
\item ID-nummer til hvert forsøg skal angives.
\item Der skal hele tiden være vist Data. De første 7 sekunder bliver vist, hvorefter den kontinuert bliver opdateret og overskriver det foregående.
\end{enumerate}
\end{enumerate}


\section{Dato: 24-09-2015}
\hrule

\textbf{Fremmødte:} Martin, Mette, Ditte, Anne, Albert og Mathias 

\textbf{Fraværende:} Ingen

\textbf{Referent:} Mathias Munk

\textbf{Dagens dagsorden:}
\newline
Formålet med mødet er, at arbejde videre med kravspecifikation og accepttest.
\begin{enumerate}
\item Ændre og tilrette kravspecifikation mht. konklusionerne fra vejledermødet.

\item Ændre og tilrette accepttest

\item LaTeX - Status på oprettelse af generelle dokumenter til rapporten?

\item Gennemgang af tidsplan og tilføjelse af interne deadlines. Følgende punkter skal have et begyndelsestidspunkt samt deadline:
\begin{enumerate}
\item Officielle deadlines
\item Definering af forsknings-case
\item Projektformulering, kravspek. og accepttest
\item Software - Koden
\item Hardware
\item Hardware og software design
\item Korrekturlæsning
\item Print
\item Flere ???
\end{enumerate}
\end{enumerate}

\textbf{Referat:}

\begin{enumerate}
\item Mette ændrer i kravspecifikationen derhjemme, når Ditte og Anne er færdige med at ændre og tilrette i Use Casene. Ditte vil sende den samlede rettelse til Peter senest søndag aften, så det kan være klar til mødet tirsdag.
\item Mette ændrer i accepttesten derhjemme, når Ditte og Anne er færdige med at ændre og tilrette i Use Casene.
\item De generelle dokumenter vil blive oprettet løbende med at vi kommer til dem.
\item Tidsplanen blev gennemgået og punkterne fra dagsordenen har fået et begyndelsestidspunkt og en deadline. 
\end{enumerate}

\section{Dato: 29-09-2015}
\hrule

\textbf{Fremmødte:} Mette, Martin, Ditte, Anne, Albert og Mathias 

\textbf{Fraværende:} Ingen

\textbf{Referent:} Mathias Munk

\textbf{Dagens dagsorden:}
\newline
Formålet med mødet er, at få færdiggjort kravspecifikation og accepttest fuldstændig, samt påbegynde software-del.
\begin{enumerate}
\item Kommentarer/spørgsmål til Kravspecifikation og accepttest

\item Kl. 13:00 - Vejledermøde med Peter, lokale 204E

\item Påbegynde programmering (se nedestående forslag)

\item Evt. start på design af GUI
\end{enumerate}

Et forslag går på at vi skal nå frem til en fælles forståelse af hvordan vi griber programmeringen an. Dette kan gøres ved at vi deler os op i mindre grupper (2-3 personer) som tildeles et kode-element, hver gruppe snakker sig så frem til hvad netop denne lille bid af koden skal kunne  og mulige måder til hvordan det kan implementeres. Det vil aktivere flere i processen og Så har vi en række forslag inden selve programmeringen påbegyndes.

\textbf{Referat:}
\begin{enumerate}
\item Peters rettelser af kravspecifikation og accepttest er kigget igennem og ordnet. De rettelser, som der er undren over er highlighted af Anne og vil tages videre til vejleder mødet kl. 13.
\item I stedet for "der udvikles en transducer" skal der skrives "der udvikles instrumentering til en udleveret transducer. 
\item Peters forslag
\begin{enumerate}
\item Dilemma omkring hvor lang tid den skal gemme. Hvad er nok tid? Evt. en rød lampe der skal blinke, så der ikke måles i lang tid.
\item Når der trykkes Gem og derefter Stop, skal der stadigvæk monotereres efter det gemte interval.
\item Gemte filers navngivning skal indeholder noget indtastet og autogeneret.
\end{enumerate}
\item Gruppen er delt op og de enkelte personer enten 1 eller 2 personer sammen, kigger på de bestemte Use Cases og tænker over hvordan det skal realiseres i Visual Studio. 
\item Der er ikke startet på design af GUI endnu. Anne vil evt. starte på det derhjemme, hvis hun ikke har andet at lave.
\end{enumerate}


\section{Dato: 06-10-2015}
\hrule

\textbf{Fremmødte:} Mette, Ditte, Anne, Martin, Albert og Mathias 

\textbf{Fraværende:}

\textbf{Referent:} Mathias Munk

\textbf{Dagens dagsorden:}
\newline
Formålet med mødet er, at gennemgå Grp. 1's arbejde samt hvordan vi kommer videre fra hvor vi står nu.
\begin{enumerate}
\item Gennemgang af Grp. 1's kravspek. og accepttest - Sørg for at have læst denne igennem hjemmefra.

\item Skal vi opdeles i en Software og Hardware gruppe? Ansvarsområder - mindre opgaver.

\item Udkast til GUI

\item Evt.? Spørgsmål til vejleder.
\end{enumerate}

\textbf{Referat:}
\newline 
\begin{enumerate}
\item Gennemgangen af Gruppe 1's kravspek. og accepttest er færdiggjort. Use Cases, ikke-funktionelle krav og accepttest er rettet i gennem samt versionshistorik, aktør-kontekst diagram og aktør beskrivelse. 
\item Gruppen er enige i at vi bliver delt op i	software og hardware ansvarsgrupper.
\begin{enumerate}
\item \textbf{Software: } Anne, Mette og Mathias.
\item \textbf{Hardware: } Ditte, Albert og Martin.
\end{enumerate}
\item Der er oprettet et google dokument hvori der er skrevet ind, hvad vores software skal indeholde. Dette gælder for GUI, Logik og Data lag samt flere software relaterede informationer.
\item Ditte sender en mail til Peter om kravspek. til hardware delen.
\end{enumerate}

\section{Dato: 20-10-2015}
\hrule

\textbf{Fremmødte:} Mette, Ditte, Anne, Martin, Albert og Mathias 

\textbf{Fraværende:}

\textbf{Referent:} Mathias Munk

\textbf{Dagens dagsorden:}
\newline
Formålet med mødet er, at de to undergrupper arbejder videre med software og hardware.
\begin{enumerate}
\item Opsamling efter ferien:
\begin{enumerate}
\item Har vi fået svar fra Peter?
\item Nogen der har arbejdet med noget i ferien?
\end{enumerate}

\item Tjekke kravspek. og accepttest igennem

\item Gennemgang af brugen af DAQ ved Brian

\item Videre arbejde med software og hardware i undergrupper

\item Til slut: Status på software og hardware udvikling
\end{enumerate}

\textbf{Referat:}
\newline 
\begin{enumerate}
\item Opsamling efter ferien:
\begin{enumerate}
\item Vi har intet svar fået fra Peter i løbet af ferien, men vi fik et i løbet af dagen.
\item Anne har startet på GUI opsætning.
\end{enumerate}
\item Alt er tjekket igennem.
\item Brians gennemgang
\item Hardware gruppen er stødt på nogle spørgsmål, som bliver sendt til Peter og gruppen får svar til mødet i morgen. Software er stadig i begynder fasen og opsætning er stadig pågående.
\end{enumerate}


\section{Dato: 03-11-2015}
\hrule

\textbf{Fremmødte:} Mette, Ditte, Anne, Martin, Albert og Mathias 

\textbf{Fraværende:}

\textbf{Referent:} Mathias Munk

\textbf{Dagens dagsorden:}
\newline
Formålet med mødet er, at det primære fokus er på udarbejdelse af ISE diagrammer. Diagrammerne vises til vejleder Peter på mødet onsdag.
\begin{enumerate}
\item Status fra Hardware-gruppe

\item Status fra Software-gruppe
\begin{enumerate}
\item Kan vi få indsendt et sinus-signal via DAQ'en og Analog Discovery?
\item Implementering af logiklag
\end{enumerate}

\item Udarbejde ISE-diagrammer
\begin{enumerate}
\item Domænemodel - Udkast er lavet
\item Overordnet sekvensdiagram
\item Detaljeret sekvensdiagram
\item Applikationsmodel
\item BDD - Udkast er lavet
\item IBD (Husk komponentliste)
\end{enumerate}
\item Tidsplan - Tilrettes?
\end{enumerate}
Næste møde er i morgen onsdag startende med vejledermøde kl. 9:00


\textbf{Referat:}
\newline 
\begin{enumerate}
\item Filter virker som det skal
\item Koden er indtil fint delt op i 3-lagsmodellen. Der er endnu ikke tilføjet filter samt grænseværdier endnu.
\begin{enumerate}
\item Det gøres med Waveforms og Analog Discovery.
\item Skal følges op på.
\end{enumerate}
\item ISE Diagrammer
\begin{enumerate}
\item Mathias sørger for at udkastet bliver kigget igennem igen af Kim Bjerge.
\item Anne, Mette og Mathias færdiggører et overordnet sekvensdiagram.
\item Følges op på.
\item Anne, Mette og Mathias tager fat på applikationsmodellerne.
\item Hardware gruppen, Ditte, Albert og Martin, tager fat på BDD SysML af hardware.
\item Hardware gruppen, Ditte, Albert og Martin, tager også fat på IBD af hardware.
\end{enumerate}
\item Intet at tage fat på.
\end{enumerate}

\section{Dato: 04-11-2015}
\hrule

\textbf{Fremmødte:} Mette, Ditte, Anne, Martin, Albert og Mathias 

\textbf{Fraværende:}

\textbf{Referent:} Mathias Munk

\textbf{Dagens dagsorden:}
Der er ingen dagsorden for dette møde. Der blev stilt spørgsmål og præsenteret hvad der var opnået indtil videre, til Peter.

\textbf{Referat:}
\newline 
\begin{enumerate}
\item \textbf{Sekvensdiagram: }In vitro skal ændres til Transducer. Dette ændrer Mette.
\item \textbf{Domæne model: }Det er svært at få færdiggjort modellen 100% uden at hvordan implementering af nulpunktsjustering og kalibrering bliver. Der skal dog ændres i nogle tekster mellem hardware-blok-forbindelserne.
\item \textbf{BDD Diagram: }Der skal tilføjes en Transducer og Interface skal ændres til et andet navn. Dette står Hardware gruppen for.
\item \textbf{IBD Diagram: }Der skal tilføjes transducer.
\end{enumerate}

\section{Dato: 10-11-2015}
\hrule

\textbf{Fremmødte:} Mette, Ditte, Anne, Martin, Albert og Mathias 

\textbf{Fraværende:}

\textbf{Referent:} Mathias Munk

\textbf{Dagens dagsorden:}

\textbf{Referat:}
\newline 