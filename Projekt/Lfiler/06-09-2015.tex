\chapter{Logbog}

\section{Dato: 06-09-2015}
\hrule
\textbf{Omhandler:} Oprettelse af Github og rapport

\textbf{Ansvarlig:} Martin

\underline{\textbf{Logbog}}


Der er nu oprettet et repo på github (Semesterprojekt3) som alle i gruppen får adgang til. Har efterfølgende oprettet et nyt projekt i latex og har tilføjet de første sektioner, såsom indholdsfortegnelse og underskrifter men har også oprettet logbog og møderef.

\section{Dato: 12-10-2015}
\hrule
\textbf{Omhandler:} GUI i software 

\textbf{Ansvarlig:} Anne

\underline{\textbf{Logbog}}

Der er lavet et GUI udkast i Visual Studios. Der er tilføjet forskellige Forms til brugervalgte instillinger. Skal der nulpunktsjusteres og/eller kalibreres.

\section{Dato: 20-10-2015}
\hrule
\textbf{Omhandler:} Hardware (forstærker og filter) og software (GUI samt oprettelse af 3-lagsmodellen) 

\textbf{Ansvarlig:} Ditte, Albert og Matin (HW) - Anne, Mette og Mathias (SW)

\underline{\textbf{Logbog}}

\textbf{Hardware:} Udregning af komponentværdier, samt finde teori bag. Opbygning af filter og forstærkning på fumlebræt. Start af test på filter og modstand. \\
\\ 
\textbf{Software:} Der er oprettet klasser til 3-lagsmodellen. Der er kigget på hvordan DAQ-klassen skal tilføjes og hvordan koden skal skrives. Der er taget udgangspunkt i et eksempel der blev fremvist for flere fra studiet, men der er kommet nogle spørgsmål til hvad der helt præcist skal trækkes videre fra eksempel.

\section{Dato: 21-10-2015}
\hrule
\textbf{Omhandler:} Hardware og oprettelse af to-do liste af rapport og dokumentation.

\textbf{Ansvarlig:} Mathias, Anne, Mette, Martin, Ditte og Albert.

\underline{\textbf{Logbog}}

\textbf{Hardware: }Byttet om på modstande i forhold til INA-114 og test af udgangssignal. 
\\
Der er udover hardware blevet oprettet to-do lister til rapport og dokumentationen, sådan at de enkelde dele af hver er skrevet ned.

\section{Dato: 27-10-2015}
\hrule
<<<<<<< HEAD
\textbf{Omhandler:} BDD Domænemodel for blodtyrks system. 

\textbf{Ansvarlig:} Mathias og Mette.

\underline{\textbf{Logbog}}

\textbf{Hardware og Software: } BDD domænemodel for blodtryks systemet er blevet lavet. Vi havde lidt problemer med, at få beskrevet Hardware blocken ordenligt i diagrammet. Kim hjalp med at få BDD'et til at være korrekt. Vi skal have møde med Peter i morgen og viser ham BDD, for at høre hvad han siger til det. 
Vi mangler at lave IBD, der skal stemme overens med BDD'et. Så mangler vi Sekvens diagram for selve softwaren (det går vi i gang med i morgen).
\\
=======
\textbf{Omhandler:} Komponentliste diagram

\textbf{Ansvarlig:} Martin, Ditte og Albert.

\underline{\textbf{Logbog}}

\textbf{Hardware: } Lavet et komponent-liste diagram i Visio Studios, og dannet en figur ud fra dette. 
\\
Vi har lavet spørgsmål til Peter i morgen for hold til Hardware delen (signal, lavpasfilter test)
>>>>>>> origin/master
