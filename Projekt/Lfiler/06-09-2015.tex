\chapter{Logbog}

\section{Dato: 06-09-2015}
\hrule
\textbf{Omhandler:} Oprettelse af Github og rapport

\textbf{Ansvarlig:} Martin

\underline{\textbf{Logbog}}


Der er nu oprettet et repo på github (Semesterprojekt3) som alle i gruppen får adgang til. Har efterfølgende oprettet et nyt projekt i latex og har tilføjet de første sektioner, såsom indholdsfortegnelse og underskrifter men har også oprettet logbog og møderef.

\section{Dato: 12-10-2015}
\hrule
\textbf{Omhandler:} GUI i software 

\textbf{Ansvarlig:} Anne

\underline{\textbf{Logbog}}

Der er lavet et GUI udkast i Visual Studios. Der er tilføjet forskellige Forms til brugervalgte instillinger. Skal der nulpunktsjusteres og/eller kalibreres.

\section{Dato: 20-10-2015}
\hrule
\textbf{Omhandler:} Hardware (forstærker og filter) og software (GUI samt oprettelse af 3-lagsmodellen) 

\textbf{Ansvarlig:} Ditte, Albert og Matin (HW) - Anne, Mette og Mathias (SW)

\underline{\textbf{Logbog}}

\textbf{Hardware:} Udregning af komponentværdier, samt finde teori bag. Opbygning af filter og forstærkning på fumlebræt. Start af test på filter og modstand. \\
\\ 
\textbf{Software:} Der er oprettet klasser til 3-lagsmodellen. Der er kigget på hvordan DAQ-klassen skal tilføjes og hvordan koden skal skrives. Der er taget udgangspunkt i et eksempel der blev fremvist for flere fra studiet, men der er kommet nogle spørgsmål til hvad der helt præcist skal trækkes videre fra eksempel.

\section{Dato: 21-10-2015}
\hrule
\textbf{Omhandler:} Hardware og oprettelse af to-do liste af rapport og dokumentation.

\textbf{Ansvarlig:} Mathias, Anne, Mette, Martin, Ditte og Albert.

\underline{\textbf{Logbog}}

\textbf{Hardware: }Byttet om på modstande i forhold til INA-114 og test af udgangssignal. 
\\
Der er udover hardware blevet oprettet to-do lister til rapport og dokumentationen, sådan at de enkelde dele af hver er skrevet ned.

\section{Dato: 27-10-2015}
\hrule
\textbf{Omhandler:} BDD Domænemodel for blodtyrks system. 

\textbf{Ansvarlig:} Mathias og Mette.

\underline{\textbf{Logbog}}

\textbf{Hardware og Software: } BDD domænemodel for blodtryks systemet er blevet lavet. Vi havde lidt problemer med, at få beskrevet Hardware blocken ordenligt i diagrammet. Kim hjalp med at få BDD'et til at være korrekt. Vi skal have møde med Peter i morgen og viser ham BDD, for at høre hvad han siger til det. 
Vi mangler at lave IBD, der skal stemme overens med BDD'et. Så mangler vi Sekvens diagram for selve softwaren (det går vi i gang med i morgen).
\\

\textbf{Omhandler:} Komponentliste diagram

\textbf{Ansvarlig:} Martin, Ditte og Albert.

\underline{\textbf{Logbog}}

\textbf{Hardware: } Lavet et komponent-liste diagram i Visio Studios, og dannet en figur ud fra dette. 
\\
Vi har lavet spørgsmål til Peter i morgen for hold til Hardware delen (signal, lavpasfilter test)
>>>>>>> origin/master

\section{Dato: 28-10-2015}
\hrule
\textbf{Omhandler:} Opdele programmeringen i projekter

\textbf{Ansvarlig:} Anne

\underline{\textbf{Logbog}}

\textbf{Software: }Vi har besluttet at opdele koden i små projekter, så hvert lag/del af programmet er i sit eget projekt. Dette er valgt da det så er lettere at arbejde flere personer med koden. Koden opbygges efter principperne i tre-lagsmodellen. Derudover kan der nu indlæses data fra DAQ'en via brug af tråde, dog udenom Logiklaget. 
\\

\section{Dato: 02-11-2015}
\hrule
\textbf{Omhandler:} Implementering af database og tilføje noter i systemkoden

\textbf{Ansvarlig:} Anne

\underline{\textbf{Logbog}}

\textbf{Software: }Database er blevet implementeret, så der i databasen skal gemmes 'Forsøgsnavn, AutogeneretNR og det ufiltrede signalet'. Logiklaget er på nuværende tidspunkt ikke implementeret helt, dette skal jeg have inputs til fra gruppen på tirsdagensmøde.  
I koden har jeg tilføjet en række kommentarer til senere brug.
\\

\section{Dato: 04-11-2015}
\hrule
\textbf{Omhandler:} Data igennem logiklag, indledninger til rapport og dokumentation. Krav til rapporten skrives. 
Oprettelse af alle manglende emner i rapporten, i Latex. 

\textbf{Ansvarlig:} Anne, Mette, Mathias og Ditte. 

\underline{\textbf{Logbog}}

\textbf{Software og dokumentation: }Der er blevet skrevet indledning til både rapporten og til dokumentationen. Krav-delen er næsten skrevet færdig, til rapporten. 
Der kan nu gå data igennem logiklaget. 
Der er blevet oprettet Latex dokumenter for hvert afsnit, vi mangler i rapporten. Så når vi skal i gang med selve skrivningen af rapport, er det let at gå ind og finde det enkelte afsnit man vil skrive. 
Det vi kiggede på til slut i går var consumer og det arbejder vi videre med senest på tirsdag. 
\\

\section{Dato: 09-11-2015}
\hrule
\textbf{Omhandler:} Udarbejde andet udkast til domænemodel og sekvensdiagrammer til alle fem Use Cases

\textbf{Ansvarlig:} Anne

\underline{\textbf{Logbog}}

\textbf{Software og dokumentation: }Med udgangspunkt i de fem Use Cases og teori for domænemodel er der blevet udarbejdet et andet udkast til domænemodellen, hvor PC bliver erstattet af System og nulpunktsjustering og kalibrering flyttes ind som en del af system istedet for at have deres egen.
Til hver Use Case er der udarbejdet et sekvensdiagram med de nødvendige metodekald for at få udført de ønskede handlinger.
\\

\section{Dato: 10-11-2015}
\hrule
\textbf{Omhandler:} Samle HW og SW diagrammer i et dokument, samt test og dokumentation af hardware.

\textbf{Ansvarlig:} Alle

\underline{\textbf{Logbog}}

\textbf{Software, hardware og dokumentation: }Først blev andet udkast til domænemodellen vendt i gruppen, og fælles løsning blev udarbejdet. Derefter blev alle sekvensdiagrammer til hver enkelt Use Case tegnet ind i Visio. Derefter blev samtlige diagrammer samlet og korte beskrivelser tilføjet. Til slut blev diagrammer + kravspecifikation sendt til Gruppe 5. 
På hardware siden er det blevet diskutteret om waveform-billeder skal placeres i selve dokumentationen eller som bilag. Konklusionen blev at er de vigtige for forståelsen af den skrevne dokumentation og test placeres billedet i dokumentationen. Herefter er test af lavpasfiltret blevet dokumenteret i Dokumentationen.
\\

\section{Dato: 13-11-2015}
\hrule
\textbf{Omhandler:} Software-kode: Implementering af Observer og strategy

\textbf{Ansvarlig:} Mathias og Anne

\underline{\textbf{Logbog}}

\textbf{Software: }Vi har oplevet store problemer med at få flyttet oprettelsen af tråde fra præsentationslaget ned i logiklaget. Dette kan løses med Observer-Strategy princippet. 
Implementeringen: Der tages udgangspunkt i et eksempel fra ITS3-undervisningen. ConsumerSubjekt placeres som en metode i logiklaget, denne skal arve fra klassen Subject. Et interface af navnet Observer oprettes, samt en metode i præsentationslaget svarende til ConsumerObserver oprettes. Vores DAQ-data flyttes nu i et array fra logiklaget til præsentationslaget, hvor metoden updateChart() findes. Dette kan lade sig gøre kan updateChart() notify hver eneste gang en ny tråd oprettes i logiklaget. 
Næste step kodemæssig er at der skal kigges på kalibrering og nulpunktsjustering.
\\

\section{Dato: 18-11-2015}
\hrule
\textbf{Omhandler:} Forsknings delen 

\textbf{Ansvarlig:} Mette

\underline{\textbf{Logbog}}

Startede på anatomi og fysiologi delen omkring faglig viden om blodtryk i går. Vil skrive generelt om blodtryk, forhøjet blodtryk og for lavt blodtryk. Hvilke sygdomme kan tilstandene føre til? Hvorfor det er vigtigt at få målt blodtryk. 
\\
\newpage
\section{Dato: 26-11-2015}
\hrule
\textbf{Omhandler:} Generelt udvikling af projektet

\textbf{Ansvarlig:} Mathias

\underline{\textbf{Logbog}}

Der er diskuteret hvad der skal gøres med hardwaret. Enigheden lød på at få eksporteret det over i Ultiboard hvor vi derefter kan få lavet et færdigt print. Dette print vil dog ikke være klar til aflevering, men inden eksamen. Indtil videre laves der et VERO Board i stedet for fumlebræt.
Projektformuleringen og afgrænsning er færdiggjort. Modultest i forhold til vandsøjlen er gjort. Anatomi er færdiggjort. Software skal stadig arbejdes videre med. I gruppen er der problemer med LateX. 
\\

\section{Dato: 02-12-2015}
\hrule
\textbf{Omhandler:} LaTex dokumenter

\textbf{Ansvarlig:} Mette og Martin. 

\underline{\textbf{Logbog}}

Der er nu blevet oprettet figurlister til både rapporten og dokumentationen. Figurlisten opdaterer selv og tilretter rækkefølgen og numre. Der er blevet lavet referenceliste i både rapporten og dokumentationen. Opdatering af referencelisten fungerer på sammen med som med figurlisten. Retter selv nummer og rækkefølgen.  
\\